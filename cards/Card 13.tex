\section{Вопрос 13. Кульминация и завершение Смутного времени (1610--1618 гг.) Исторические последствия Смуты.}

\subsection{Кульминация Смутного времени:}

На момент 1610 г. Юго-Запад страны находится под контролем польского короля Сигизмунда III, Северо-Запад захвачен шведскими войсками, сама Москва оккупирована польским гарнизоном, который фактически ею управлял. Страна на грани утраты национальной независимости.

Разорение русской земли вызвало широкий подъём патриотического движения. Начали образовываться народные ополчения. Весной 1611 г. собирается первое народное ополчение из самых разных слоёв населения во главе с Прокопием Ляпуновым, князем Трубецким и Заруцким. Оно двигалось к Москве и начало осаждать город, но после убийства Ляпунова распалось.

Осенью 1611 г. в Нижнем-Новгороде возникает второе народное ополчение во главе с земским старостой Кузьмой Мининым и князем Дмитрием Пожарским. Через Ярославль оно движется к Москве. В 1612 г. ополчение осаждает сначала Китай-Город, а потом разбивает польский гарнизон. Интервенты вынуждены капитулировать.

Освобождение столицы позволило восстановить государственную власть. В январе 1613 г. собирается земский собор, на котором на царствование избирается \textbf{Михаил Романов}. Перед ним в первую очередь стояла задача стабилизировать ситуацию в стране. 

После неудачной попытки Швеции овладеть Псковом, она идёт на мирные переговоры и в 1617 г. заключается Столбовский мир. По нему Россия возвращает Новгородскую землю, но теряет территорию Карелии и единственный выход к Балтийскому морю.

В 1618 г. с Речью Посполитой заключается Деулинское перемирие. Так завершается Смутное время.

\subsection{Исторические последствия Смуты:}

\begin{enumerate}
    \item{ Хозяйственное разорение страны }
    \item{ Утрата территорий и внешнеполитического влияния }
    \item{ Стремление к изоляции }
    \item{ Установление новой династии Романовых }
    \item{ Сохранение государства }
\end{enumerate}
