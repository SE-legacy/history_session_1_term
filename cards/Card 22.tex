\section{Вопрос 22. Внешняя политика России в первой четверти XIX в.}

\subsection{Участие России в коалиционных войнах против Франции. Тильзитский Мир.}

1805 г. "--- третья коалиция против Франции.

\textbf{Состав:} Россия, Австрия и Англия. 

Быстро потерпела поражение в ноябре 1805 под Аустерлицем. Австрия выходит из войны, развал Коалиции.

1806 г. "--- четвертая коалиция против Франции.

\textbf{Состав:} Россия, Пруссия, Англия и Швеция. 

Вывод из войны Пруссии в течении недель. Июнь 1807 "--- поражение России под Фридляндом, заставившее подписать Александра Тильзитский мирный и союзный договор.


Условия Тильзитского Мира:

\begin{enumerate}
    \item{ Создание Герцогства Варшавского }
    \item{ Присоединение России к континентальной блокаде против Англии (невыгодно для нашей экономики) }
\end{enumerate}

\subsection{Войны России с Ираном, Турцией и Швецией. Территориальный Рост империи.}

\subsubsection{Русско"=турецкая война 1806--1812 гг.}

\textbf{Причины:}

\begin{enumerate}
    \item{ Стремление России усилить свои позиции на Ближнем Востоке. }
    \item{ Стремление расстроить реваншистские планы Турции, не оставлявшей надежду вернуть часть Черноморского побережья. }
    \item{ Поддержка восставших против султана сербов. }
\end{enumerate}

Война проходит с переменным успехом, но после назначения Кутузова командующим Дунайской армией в 1811 г. Россия наносит сокрушительное поражение Турции и заключает Бухарестский мирный договор в мае 1812 г.

\textbf{Условия:}

\begin{enumerate}
    \item{ Переход Бессарабии и кавказского побережья России. }
    \item{ Автономия Дунайских княжеств и Сербии }
\end{enumerate}

\subsubsection{Русско"=иранская война 1804--1813 гг.}

\textbf{Причины:}

\begin{enumerate}
    \item{ Иран пытался предотвратить российское проникновение в Закавказье }
    \item{ Грузия уже в тот момент вошла в состав России, чтобы избежать захват мусульманскими странами. }
\end{enumerate}

\textbf{Итог:}

Поражение Ирана и заключение Гюлистанского мирного договора 1813 г.

\textbf{Условия:}

\begin{enumerate}
    \item{ Иран признаёт русское владычество над большей частью Закавказья, Дагестаном и западным побережьем Каспийского моря }
    \item{ Лишь Россия имеет флот на Каспийском море. }
\end{enumerate}

\subsubsection{Русско"=шведская война 1808--1809 гг.}

\textbf{Итог:}

Завоевание Россией Финляндии. Финляндия входит в состав Российской Империи как автономное государство.

\subsubsection{Отечественная война 1812 Года. Цели Воюющих сторон.}

\textbf{Причины:}

\begin{enumerate}
    \item{ Стремление Наполеона к мировому господству }
    \item{ Несмотря на Тильзидский мир, Россия продолжала противодействовать расширению наполеоновской агрессии. Так например было нарушено соблюдение Континентальной блокады Англии }
\end{enumerate}

\subsection{Отечественная война. М.Б. Барклай-де-Толли и его роль на начальном этапе войны. Назначение главнокомандующим Кутузова.}

Первая армия Барклая-де-Толли насчитывала 120 тыс человек, что являлось крупнейшей русской армией. Барклай, осознавая важность сохранения армии, отступал, заманивал французов вглубь страны, где соединившись со второй армией Багратиона в Смоленске, разрушил план Наполеона разбить армии поодиночке. Армии начали готовиться к обороне города.

Однако генерального сражения не последовало и Барклай сдал город. Александр снял Барклая с должности главнокомандующего и назначил на его место Кутузова 8 августа. 

Кутузов продолжил тактику Барклая избегания генерального сражения, но вся старана ждала решительных действий, и Кутузов начал поиск подходящего места.

Также именно Барклай подал мысль о создании «летучих отрядов» или же партизан.

\subsubsection{Партизанское движение.}

До оставления Москвы партизанские отряды возникали стихийно и состояли преимущественно из крестьян. Однако уже тогда Барклай де Толли подал мысль о создании «летучих отрядов» из состава регулярных войск.

После оставления Москвы партизанское движение приняло широкий размах. Для партизанской деятельности было выделено 48 полков, им поставлялось снаряжение, некоторым полкам даже артиллерия.

Они нарушали коммуникации противника, выполняли роль разведки, блокировали отступавшую французскую армию, лишая ее фуража и продовольствия.


\subsubsection{Отечественная война. Бородино.}

Место для генеральной битвы было найдена у села Бородино. Кутузов выбрал оборонительную тактику. Левый фланг защищала армия П. И. Багратиона, прикрытая искусственными земляными укреплениями "--- флешами. В центре был насыпан земляной курган, где расположились артиллерия и войска генерала Н. Н. Раевского. Армия М. Б. Барклая де Толли находилась на правом фланге.

Соотношения сторон было почти равным: у французов "--- 130 тыс. человек при 587 орудиях, у русских "--- 110 тыс. человек регулярных сил, около 40 тыс. ополченцев и казаков при 640 орудиях.

Атака французов началась рано утром 26 августа и продолжалось более 12 часов. Обе стороны понесли серьезные потери. Погибло 29 генералов, в том числе и Багратион. Началось отступление к Москве.

\subsubsection{Изгнание французов из России.}

1 Сентября в деревне Фили прошел военный совет, на котором Кутузов принял решение оставить Москву.

2 сентября Французы вошли в город.

Далее, Кутузов совершил Тарутинский марш-маневр. Отступая из Москвы по Рязанской дороге, армия круто повернула к югу и в районе Красной Пахры вышла на старую Калужскую дорогу.

Итоги такого маневра:

\begin{enumerate}
    \item{ предотвратил захват французами Калужской и Тульской губерний, где были собраны боеприпасы и продовольствие. }
    \item{ М. И. Кутузову удалось оторваться от армии Наполеона и дать армии передохнуть. }
\end{enumerate}

Наполеон, поняв, что Александр I не будет заключать мир начал отступление из Москвы 7 октября.

Двинув войска на Калужскую дорогу, Наполеону пришлось дать бой у г. Малоярославец. ни та, ни другая сторона не добилась решительной победы. Однако французы были остановлены и вынуждены отступать по ими же разоренной Смоленской дороге.

Бегство французов ускоряло партизанское движение и наступательные действия русских. Серьезный урон был нанесен французам под г. Красным в начале ноября, когда из 50 тыс. человек отступающей армии более половины было взято в плен или пало в бою.
При переправе через р. Березина Наполеон бросил армию и тайно поехал в Париж.

Манифест царя от 25 декабря 1812 г. ознаменовал завершение Отечественной войны.

\subsection{Россия в коалиционных войнах против Франции в 1813--1814 гг.}

Январь 1813 г. "--- русские войска вступили на территорию Польши и Пруссии. Пруссия заключила союз с Россией. К ним примкнули Австрия, Англия и Швеция.

Октябрь 1813 г. "--- "Битва Народов" под Лейпцигом.

Март 1814 г. "--- Наполеон сослан на о. Эльба в Средиземном море.

\subsection{Венский конгресс.}

В сентябре 1814 г. "--- июне 1815 г. державы-победительницы решали вопрос о послевоенном устройстве Европы.

Наполеон сбегает с о. Эльба и СНОВА побежден, под Ватерлоо. Сослан в 1815 на о. Святой Елены в атлантическом океане.

\textbf{Результаты Венского конгресса:}

\begin{enumerate}
    \item{ Восстановление династий во Франции, Испании, Италии и др. }
    \item{ Из польских земель создано Царство Польское, вошедшее в состав России }
\end{enumerate}

Март 1815 г. "--- Четверной Союз где Россия, Англия, Австрия и Пруссия следят за исполнением конгресса.

\subsection{Россия во главе священного союза.}

В сентябре 1815 г. российский император Александр I, австрийский император Франц и прусский король Фридрих Вильгельм III подписали Акт об образовании Священного союза. Текст Александра имел религиозно-мистический характер.

\textbf{Цели союза:}

\begin{enumerate}
    \item{ Помощь друг"=другу }
    \item{ Поддержка старых династий  }
    \item{ Борьба с революционным движением в Европе. }
\end{enumerate}

В 20"=е годы противодействие революционным движениям: подавление революций в Италии и Испании.
