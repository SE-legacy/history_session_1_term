\section{Вопрос 15. Внешняя политика России в эпоху первых Романовых.}

\subsection{Основными направлениями внешней политики в эпоху первых Романовых были:}

\begin{enumerate}
    \item{ Достижение выхода к Балтийскому морю }
    \item{ Предотвращение набегов крымского хана }
    \item{ Возвращение утерянных в эпоху Смуты земель }
\end{enumerate}

\subsection{Внешняя политика:}

\begin{enumerate}
    \item{ Смоленская война (1632--1634 гг.). Смоленск был взят в осаду, но войти в сам город не получилось. }
    \item{ В 1637 г. по собственной инициативе донские  казаки берут крепость Азов. Созванный земский собор понимал выгоды от присоединения крепости (выход в Азовское и Черное море), но был вынужден отказаться ввиду неизбежной войны с Османской империей }
    \item{ Продолжает осваиваться Сибирь:

        \begin{enumerate}
            \item{ Василий Поярков открывает побережье Охотского моря }
            \item{ Семён Дежнёв открывает Берингов пролив }
            \item{ В 1689 г. заключается первый договор с Китаем "--- Нерченский договор, по которому устанавливаются границы с Китаем.  }
        \end{enumerate}
    }
    \item{ В 1648 г. начинается освободительное восстание Богдана Хмельницкого против Речи Посполитой. Бои ведутся до 1653 г. Восставшие обращаются за помощь к России. На земском соборе в 1653 г. территория Украины включается в состав России. Начинается Русско-Польская война (1654--1667 г.), по результатам которой в 1667 г. заключается Андрусовское перемирие. Возвращён Смоленск и присоединено правобережье Днепра. Украина в составе России получает автономию. }
\end{enumerate}
