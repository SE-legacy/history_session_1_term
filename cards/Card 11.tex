\section{Вопрос 11. Внешняя политика Русского государства в XVI в.}

\subsection{Направления внешней политики:}

\begin{enumerate}
    \item{ Северо"=Запад: Шведское королевство }
    \item{ Запад: Литовское княжество, Ливонский орден }
    \item{ Юг: Крымское и Астраханское ханства }
    \item{ Восток: Сибирское ханство }
\end{enumerate}

\subsection{Основной целью внешней политики Ивана IV являлось присоединение новых земель.}

\begin{enumerate}
    \item{ 1552 г. "--- взятие Казани }
    \item{ 1556 г. "--- взятие Астрахани (без боя) }
\end{enumerate} 

\subsection{<<Внешние>> территории:}

\begin{enumerate}
    \item{ 1558 г. "--- Начало Ливонской войны (1558--1583 гг.):
        \begin{enumerate}
            \item{ Основной целью войны было стремление заполучить выход к Балтийскому морю }
            \item{ В 1561 г. к Ливонскому ордену присоединяется Литовское княжество. Несмотря на успешное начало войны, Россия не смогла развить успех. }
            \item{ В 1569 г. происходит объединение Польши и Литовского княжества в одно государство "--- Речь Посполитая. Это сказалось на положении России в худшую сторону. Кроме того, в 1571 г. к войне присоединяется Шведское королевство.  }
            \item{ В 1583 году в результате безуспешной осады Пскова был заключён мир. }
        \end{enumerate}
        }
    \item{ 1581 г. "--- война за присоединение территорий Сибирского ханства под предводительством Ермака. }
    \item{ 1590--1593 гг. "--- Русско-Шведская война. Закончилась подписанием Тявзинского (тейсинского) мира. }
\end{enumerate}

