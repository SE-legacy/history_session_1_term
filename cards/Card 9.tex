\section{Вопрос 9. Объединение русских земель вокруг Москвы в конце XIII"="=середине XV в.}

После смерти Василия II к власти приходит его сын "--- Иван III (1462--1505 гг.):

\subsection{Иван III (1462--1505 гг.):}

\begin{enumerate}
    \item{ Присоединяет Ярославское (1463) и Ростовское (1472) княжества к Москве }
    \item{ Стремился присоединить Новгород. Так, в 1471 г. Произошла битва при Шелони, в которой новгородская армия потерпела сокрушительное поражение. Окончательное присоединение произошло в 1478 г. }
    \item{ В 1485 г. В результате осады Тверь так же была присоединена }
    \item{ Происходит две Русско"=Литовские войны (1494, 1503 г.) в результате которых были присоединены колоссальные территории }
    \item{ При Иване III в документах начинает фигурировать название «Росия» или «Россия». }
    \item{ В конце 1370"=х отказывается выплачивать дань. Ордынский хан Ахмат вместе с Литовским князем выступает против Ивана III и крымского хана. Происходит стояние на реке Угре (1480 г.). Хан Ахмат уходит без боя. Зависимость от Орды спадает. }
    \item{ После падения Византии Русь становится её преемницей, заимствует христианские ритуалы и символ "--- двуглавого орла. }
    \item{ Формируется аппарат управления, состоящий из великого князя, боярской думы и наместников. }
    \item{ В 1497 г. составляется «Судебник», взамен устаревшей «Русской правды», по которому регулируется жизнь всех слоёв населения, в том числе и сформировавшегося слоя крестьян. Там же ограничивается их свобода перехода. }
    \item{ В грамотах начинает фигурировать титул «князя всея Руси» и «самодержца» (т.е. автократа) }
\end{enumerate}

\subsection{Василий III (1505--1533 гг.) "--- сын Ивана III:}

\begin{enumerate}
    \item{ Присоединяет Псков в 1510 г., Смоленск в 1514 г., Переяславль"=Рязанский в 1521 г. Заканчивает процесс собирания Русских земель. }
    \item{ Женился на Елене Глинской. Наследник родился незадолго до смерти князя. }
\end{enumerate}
