\section{Вопрос 5. Русские земли во второй половине XI "--- первой трети XII вв. Причины и обстоятельства распада государства. Общие условия развития русских княжеств.}

После смерти Ярослава Владимировича (Мудрого) началась политическая раздробленность, которая продлилась с середины XI до начала XII века.

\subsection{Причины:}

\begin{enumerate}
    \item{ Этническая неоднородность (На больших землях находилось множество разных народов, которые стремились объединиться внутри своей этнической группы)}
    \item{ Рост княжеского рода. Лествичная система наследования. (Великий князь распределял земли между своими сыновьями по старшинству. А те, в свою очередь распределяли полученные от отца земли между своими сыновьями. Происходило дробление земель на всё более мелкие. Возникали усобицы.)}
    \item{ Быстрый рост городов}
\end{enumerate}

Киевляне были недовольны политикой Изьяслава и подняли  восстание 1068 года, после которых Изяслав повёл на Русь поляков, а Святослав и Всеволод выступили в защиту Киева.

В 1071 году Всеславу удалось вернуться в Полоцк, после чего братья заподозрили Изяслава в союзе с ним и изгнали с престола. 

После княжил Святослав, а затем Всеволод в 1076г.

В 1097г. сьехались 6 князей, дабы обьединиться в борьбе с половцами и заключили договор, чтобы не допускать междоусобных распрей.

После сьезда каждому князю дали свои земли(Земли Ярика Мудрого), таким образом Русское княжество стало совокупностью отдельных отчин.

После смерти в 1113 году киевского князя Святополка Изяславича в Киеве вспыхнуло народное восстание, основной причиной которого была финансовая политика администрации Святополка, в частности, введённый им соляной налог и Киевское боярство призвало на княжение Владимира Мономаха "--- сына Всеволода.

\subsection{Общие условия развития русских княжеств:}

В это время происходит рост числа боярских вотчин. Развитие этой системы привело к усилению самостоятельности бояр, уменьшению их зависимости от князя.

Из"=за раздробленности русских земель у каждого из них появилась самостоятельная внешняя политика. Существуют тексты документов, которые заключали отдельные русские земли.

Развивалось три типа политического устройства княжеств:

\begin{enumerate}
    \item{ Княжеская монархия (Единовластие князя, совещательная ф-ция бояр, не развитое вече)}
    \item{ Боярская республика (Вече "--- главный политический орган. Такая форма была лишь в Новгородском княжестве, и в XIV веке появилась в Пскове)}
    \item{ Конфликтное (Постоянные конфликты князя и боярства)}
\end{enumerate}
