\section{Вопрос 4. Древнерусское государство при князьях Владимире Святославиче и Ярославе Мудром.}

\subsection{Владимир Святославич (980--1015 гг.)}

\begin{enumerate}
    \item{ Сохранял контроль над племенами и укреплял границы}
    \item{ 988 г. Крещение Руси:

        \begin{enumerate}
            \item{ Ввиду ликвидации племенных союзов возникла структура единого государства. В идеологической сфере потребовалась религия на замену языческих культов, могущие служить сепаратистским тенденциям.}
            \item{ Перед этим Владимир предпринимает попытку создать общерусский языческий пантеон во главе с Перуном, но эта акция не увенчалась успехом.}
            \item{ Выбор новой религии стоял между православием, католицизмом, мусульманством и иудаизмом. Выбор пал в сторону известного на руси православия.}
        \end{enumerate}

        \textbf{ПОСЛЕДСТВИЯ ПРИНЯТИЯ ХРИСТИАНСТВА:}

        \begin{enumerate}
            \item{ Началась этническая консолидация (Объединение разрозненных языческих племен)}
            \item{ Интеграция в семью христианских народов}
            \item{ Создание структуры православной церкви}
            \item{ Развитие культуры (письменность и книжность)}
            \item{ Появление валюты}
        \end{enumerate}
    }
    \item{ Его эпоха являлась особенно плодотворной в формировании былинного эпоса. (Добрыня Никитич, прототипом которого являлся дядя по матери Добрыня, ставший советником и наставником князя в молодости, и Илья муромец)}
    \item{ Распространяются славянские азбуки "--- кирилица и глаголица. Первым памятником славянской письменности являлся русско-византийский договор 911 г.}
    \item{ После смерти начинается междоусобная война за престол}
\end{enumerate}

\subsection{Ярослав Владимирович (Мудрый) (1019--1054 гг.)}

\begin{enumerate}
    \item{ МЕЖДОУСОБНАЯ ВОЙНА ЗА ПРЕСТОЛ:
        \begin{enumerate}
            \item{ Являлся Новгородским князем, правившим на момент смерти Владимира. Выступил против Святополка, который захватил престол в Киеве. Разгромил, Святополк бежал в Польшу, а затем при поддержке польских войск вернулся и захватил Новгород. Ярослав, вооружившись поддержкой новгородцев, нанял варяжскую дружину и ОПЯТЬ разгромил Святополка. На этот раз окончательно}
            \item{ Воевал с племянником, Брячиславом Изяславичем, правившим в Полоцке, и братом, Мстиславом Владимировичем "--- тмутаракаиский князь. Ярослав стал единственным правителем лишь после смерти последнего в 1036 г.}
        \end{enumerate}
    }
    \item{ Поддержка религии (Софийский собор в Киеве)}
    \item{ Создан первый свод законов "--- «Русской правды»}
    \item{ Устранена опасность со стороны печенегов в 1036 г.}
\end{enumerate}
