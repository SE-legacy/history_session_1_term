\section{Вопрос 12. Предпосылки Смутного времени. Начальный период Смуты (1604--1610 гг.)}

\subsection{Предпосылки Смутного времени:}

\begin{enumerate}
    \item{ Хозяйственный кризис (Вызван затяжной Ливонской войной и губительной опричниной). Неурожай и голод в 1601--1603 гг. }
    \item{ Закрепощение крестьян }
    \item{ Политическая активность дворянства и казачества. Местничество }
    \item{ Династический кризис (прерывание династии Рюриковичей) }
\end{enumerate}

\subsection{Начальный период Смуты (1604--1610 гг.)}

\begin{enumerate}
    \item{ Правление Бориса Годунова было кризисным временем: голод, бунты, тяжёлая экономическая ситуация "--- всё это привело к падению авторитета Годунова. На этом фоне при поддержке польского короля Сигизмунда III появляется фигура чудом спасшегося царевича Дмитрия, сына Ивана IV, истинного наследника престола. Под этой фигурой скрывался простой галичский дворянин Григорий Отрепьев, так же известный как Лжедмитрий I.  }
    \item{ В 1604 г. Лжедмитрий вторгается в Юго-Западные регионы России, получает поддержку и движется к Москве. 21 января 1605 г. около села Добрыничи армия Лжедмитрия полностью разгромлена царским войском. Лжедмитрий чудом сбегает. }
    \item{ Неожиданно 13 апреля 1605 г. Умирает Борис Годунов. Боярство не признаёт сына Фёдора Годунова и 1 июня Москва присягает Лжедмитрию I }
    \item{ Григорий Отрепьев не признавал русских обычаев и традиций из-за чего стремительно терял авторитет. Последней каплей стала свадьба с Мариной Мнишек "--- католичкой. Во главе с Василием Шуйским возникает заговор, Лжедмитрий I убит 17 мая 1606 г. }
    \item{ Власть переходит к Василию Шуйскому вплоть до окончания начального этапа Смуты (до 1610 г.) }
    \item{ Но слухи о спасении царевича Дмитрия не утихают. Летом 1606 г. в городе Путивль возникает движение во главе с Иваном Болотниковым, к которому присоединяются не только крестьяне и посадские люди, но и служилые дворяне. Москву восставшим взять не удалось, они были разбиты под селом Коломенским, а позже ликвидированы, отступив в Тулу. }
    \item{ Третий самозванец объявляется летом 1607 г. из Речи Посполитой. К нему на сторону переходят многие оставшиеся болотниковцы и сторонники Лжедмитрия I. Летом 1608 г. Войско Лжедмитрия II укрепляется в подмосковном селе Тушино (за что тот получает прозвище тушинский вор). Некоторая знать переходит от Шуйского на сторону Лжедмитрия II. С 1608 г. в стране возникает двоецарствие (Шуйский в Москве, Лжедмитрий в Тушино). }
    \item{ Василий Шуйский, обеспокоенный ситуацией в стране обращается за помощью к Шведскому королю, предложив территории Карелии. Из"=за этого Речь Посполитая начинает открытую интервенцию против России. }
    \item{ На этом фоне в Москве в 1610 г. созревает бунт против Шуйского и его свергают. Власть переходит к правительству из 7 бояр (семибоярщина), которые предлагают на русский престол польского королевича Владислава.  }
    \item{ Тем временем внезапно начинается Шведская интервенция. Шведская армия захватывает Новгород }
\end{enumerate} 
