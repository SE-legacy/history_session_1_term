\section{Вопрос 16. Реформы Петра I.}

Петр I (1682--1725 гг.)

\subsection{Азовские походы (1695--1696 гг.):}

\begin{enumerate}
    \item{ 1695г. "--- осада (неудачная, Петр усвоил урок и строил новые верфи и корабли) }
    \item{ 1696г. "--- осада (удачная) }
\end{enumerate}

Многие реформы Петра I стали следствием Великого посольства (1697--1698 гг.), в котором царь вместе со своими приближёнными изучал зарубежные ремесла, быт, культуру, воинские и политические порядки.

\subsection{Реформы:}

\subsubsection{Административные:}

\begin{enumerate}
    \item{ Местное самоуправление 
        \begin{enumerate}
            \item { Губернская реформа
                \begin{enumerate}
                    \item { 1708 г.- деление территории на 8 губерний }
                    \item { 1710 г.- губернии дробятся на провинции и уезды }
                \end{enumerate}
            }
            \item{ Городская реформа
                \begin{enumerate}
                    \item{ 1699 г.- указ об «Об учреждении Бурмистерской Палаты» (позже Ратуша). Функции Ратуши: сбор налогов, суд, надзор за исполнением приказов}
                    \item{ 1720 г.- создание Главного Магистрата и изданием регламента к нему, вводилось разделение городов на разряды }
                \end{enumerate}
            }
        \end{enumerate}
    }
    \item {Центральное самоуправление
        \begin{enumerate}
            \item{ Образование Ближней Канцелярии
                \begin{enumerate}
                    \item{ 1699 г.- Ближняя канцелярия, а с 1711 г. Сенат – высшее правительственное учреждение с судебными, административными и, иногда, законодательными обязанностями}
                    \item{ Контролировал работу Сената: генерал-прокурор}
                \end{enumerate}
            }
            \item{ Канцелярии => Коллегии
                \begin{enumerate}
                    \item{ 1718 – 1720 гг.- вместо канцелярий и приказов появились коллегии }
                \end{enumerate}
            }
        \end{enumerate}
    }
\end{enumerate}

\subsubsection{Экономические:}

\begin{enumerate}
    \item{ Торговая реформа
        \begin{enumerate}
            \item{ 1724 г.- таможенный тариф }
            \item{ Налог иностранным торговцам: чем дальше, вглубь страны, тем больше }
        \end{enumerate}
    }
    \item { Денежная реформа
        \begin{enumerate}
            \item{ 1700 г.- новые монеты, меньше вес серебряных монет и т. д. }
        \end{enumerate}
    }
    \item {Промышленные реформы
        \begin{enumerate}
            \item{ Создание мануфактур (200~ по итогу) }
            \item{ 1703 г.- указ о приписных крестьянах (их приписывали к мануфактурам для работы за счет государства)}
            \item{ 1721 г.- указ о посессионных крестьян (возможность владельцам мануфактур покупать крестьян)}
        \end{enumerate}
    }
\end{enumerate}

\subsubsection{Социальные:}

\begin{enumerate}
    \item{ Реформа образования: (Организация различных специализированных школ (инженерной, горной, артеллиристской, медицинской и т.д.) детей дворян отправляли заграницу, а из Европы приглашались ученые и инженеры, которых обязывали обучать наиболее способных людей на производстве).}
    \item{ Развитие типографий: (Обеспечение методическими материалами всех учебных заведений, которые создавал Петр I. К 1714 году Петербург обзавелся собственной типографией, а книгопечатание лишь набирало обороты. С 1708 по 1725 год было напечатано книг больше, чем в предыдущие полтора века)}
    \item{ Развитие здравоохранения и медицины: (1706 г.- Основание Московского госпиталя. 1714 г.- Для обеспечения государственных аптек был основан город на Аптекарском острове)}
    \item{ 1724 г.- указ об учреждении Академии Наук и художеств. Для работы в новом учреждении были приглашены иностранные специалисты и вплоть до 1746 года большая часть академиков была иностранцами.}
\end{enumerate}

\subsubsection{Военные:}

\begin{enumerate}
    \item{ 1705г.- Введена рекрутская система комплектования армии. Полевые и гарнизонные войска набираются из числа крестьян, офицерский корпус – из числа дворян. }
    \item{ Создание школ для подготовки офицерских кадров }
    \item{ Новые воинские уставы }
    \item{ Вводится единая форма }
    \item{ Повышение в чине за воинские отличия }
    \item{ Военно-морской флот }
    \item{ Перевооружение армии (Гладкоствольное оружие на нарезное, винтовки со штыком, гранаты, мортиры) }
\end{enumerate}

\subsubsection{Культурные:}

\begin{enumerate}
    \item{ 1705г.- указ брить бороды (так же налог на бороды)}
    \item{ Новое летоисчисление и новый календарь (Согласно указу, после 31 декабря 1707 года наступит 1 января 1700 года)}
    \item{ 1714 г.- Создание кунтскамеры}
    \item{ 1718 г.- Создание ассамблей}
\end{enumerate}

\subsubsection{Сословные}

\begin{enumerate}
    \item{ 1722 г.- «Табель о рангах» (по нему служба делится на гражданскую, военную и придворную) }
\end{enumerate}

\subsubsection{Церковная:}

\begin{enumerate}
    \item{ 1721 г.- «Духовный регламент» (по нему церковь теряет автономию и подчиняется государству. Учреждена специальная коллегия – Святейший правительствующий Синод. В его ведении находились все церковные дела) }
\end{enumerate}
