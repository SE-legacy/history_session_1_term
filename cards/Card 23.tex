\section{Вопрос 23. Движение декабристов.}

\subsection{Декабристское Движение: истоки и идеология.}

Причины подъема общественного движения:

\begin{enumerate}
    \item{ Основная "--- отставание России от западно-европейских держав }
    \item{ Народные волнения }
\end{enumerate}

Причины декабристского движения:

\begin{enumerate}
    \item{ понимание лучшими представителями дворянства, что сохранение крепостничества и самодержавия гибельно для дальнейшей судьбы страны. }
    \item{ Разочарование в Александре I }
    \item{ Знакомство с Европейским бытом во время пребывания Русской армии во франции 1813-1815 г. }
\end{enumerate}

Борясь против самодержавия и крепостничества, революционеры неосознанно отстаивали буржуазный путь развития. Поэтому их движение имело объективно буржуазный характер.

\subsection{Ранние Декабристские организации.}

Самые первые (не предпринимали активных действий, но развивали взгляды её членов):

\begin{enumerate}
    \item{1811--1812 "Чока" Н. Н. Муравьева}
    \item{1814 "Священная Артель" Н. Н. Муравьева}
    \item{"Орден Русских рыцарей" М. Ф. Орлова}
\end{enumerate}

Февраль 1816 г. "--- "Союз Спасения". Его основали: П. И. Пестель, А. Н. Муравьев. С. П. Трубецкой. К ним присоединились К. Ф. Рылеев, И. Д. Якушкин, М. С. Лунин. С. И. Муравьев-Апостол и др.

«Союз спасения» "--- первая русская политическая организация, имевшая революционную программу и устав "--- «Статут».

\textbf{Цели и идеи:} ликвидация крепостного права и уничтожение самодержавия. Необходимость введения конституции.

Январь 1818 г. "--- «Союз благоденствия». насчитывал 200 чел. Организация получила довольно четкую структуру. Были избраны Коренная управа — общий руководящий орган "--- и Совет (Дума), обладавший исполнительной властью. Местные организации «Союза благоденствия» появились в Петербурге, Москве, Тульчине, Кишиневе, Тамбове, Нижнем Новгороде. 

Программа и устав союза назывались «Зеленая книга». состояла из двух частей:

\begin{enumerate}
    \item{Предназначенная для всех членов общества содержала легальные формы деятельности общества.}
    \item{Содержащая идеи о свержении самодержавия, ликвидации крепостничества, введении конституционного правления и т.д}
\end{enumerate}

Общество распущено в начале 1821 г. из-за внутренних разногласий.

\subsection{Южное и северное общество. "Русская правда" Пестеля и "Конституция" Н.М. Муравьева.}

В марте 1821 г. на Украине было образовано Южное общество. Его создателем и руководителем стал П. И. Пестель, убежденный республиканец. 

В 1822 г. в Петербурге было образовано Северное общество. Его признанными лидерами стали Н. М. Муравьев, К. Ф. Рылеев, С. П. Трубецкой, М. С. Лунин.

Основными обсуждавшимися проектами стали «Конституция» Н. М. Муравьева и «Русская правда» П. И. Пестеля. «Конституция» отражала взгляды умеренной части декабристов, «Русская правда» — радикальной.

Отличия "Конституции" Н.М. Муравьева:

\begin{enumerate}
    \item{Конституционная монархия}
    \item{При освобождении крестьян выдать по 2 десятины на двор (ОЧЕНЬ МАЛО)}
    \item{Федеративное территориальное устройство.}
\end{enumerate}

Отличия "Русской правды" Пестеля:

\begin{enumerate}
    \item{Парламентская республика с президентской формой правления.}
    \item{При освобождении крестьян создание общественного фонда земли для выдачи всем желающим прожиточного минимума.}
    \item{Унитарное государство.}
\end{enumerate}

Оба конституционных проекта касались и других сторон социально-политической системы России. Они предусматривали введение широких демократических гражданских свобод, отмену сословных привилегий, значительное облегчение военной службы солдат.

\subsection{Восстание на Сенатской площади и восстание Черниговского полка.}

После смерти царя Александра I в течение месяца в стране сложилась необычная ситуация — междуцарствие. Не зная об отречении Константина, высшие государственные чиновники и армия присягнули ему. На 14 декабря была назначена переприсяга членами Сената Николаю. Руководители Северного общества решили, что смена императоров и некоторая неопределенность ситуации с престолонаследием создали благоприятный момент для выступления.

Заговорщики хотели принудить Сенат принять их новый программный документ — «Манифест к русскому народу» — и вместо присяги императору провозгласить переход к конституционному правлению.

Рано утром 14 декабря 1825 г. наиболее активные члены Северного общества начали агитацию в войсках Петербурга. Около 3 тыс солдат вышло на площадь. Выяснилось, что Сенат уже присягнул императору Николаю I и сенаторы разошлись по домам. Предъявить «Манифест» было некому. С. П. Трубецкой, назначенный диктатором восстания, на площадь не явился. Восставшие оказались без руководства и обрекли себя на бессмысленную тактику выжидания.

Тем временем Николай собрал на площади верные ему части и решительно ими воспользовался. Артиллерийская картечь рассеяла Ряды восставших.

29 декабря 1825 г. С. И. Муравьев-Апостол и М. П. Бестужев-Рюмин подняли восстание Черниговского полка. Изначально оно было обречено на поражение. 3 января 1826 г. полк был окружен правительственными войсками и расстрелян картечью.

\subsection{Итоги и значение Декабристского движения.}

\textbf{Итоги:} 289 человек были признаны виновными. Николай I принял решение сурово покарать восставших. Пять человек были повешены. Остальных, разделив по степени виновности на несколько разрядов, сослали на каторжные работы, на поселение в Сибирь, разжаловали в солдаты и перевели на Кавказ в действующую армию.

\textbf{Историческое значение:} Они разработали первую революционную программу и план будущего устройства страны. Впервые была совершена практическая попытка изменить социально-политическую систему России. Идеи и деятельность декабристов оказали существенное влияние на следующие поколения общественных деятелей.
