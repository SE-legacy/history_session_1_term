\section{Вопрос 20. Внешняя политика России в 1762--1801 гг.}

\subsection{Русско"=турецкая война 1768--1774 гг.}

\textbf{Причины:}

\begin{itemize}
    \item{ Стремление России получить доступ к южным морям. Считалось, что это откроет новые рынки сбыта, через Чёрное море даст возможность развивать отношения со странами южной Европы и Ближнего Востока. }
    \item{ Систематическое вмешательство России в дела Речи Посполитой, угрожавшее интересам Османской империи }
    \item{ Существование франко"=османского союза, стремление Франции ограничить влияние России в Европе }
\end{itemize}

Главные события разворачиваются в Дунайских княжествах "--- Валахии и Молдавии.  

Русские войска занимают пункты, берут крепости \ldots

В это же время российский флот должен начать действия против турецкого, но на Чёрном море его не было. 

Балтийский флот вступил в сражение с турецким в бухте Чесма (Чесменское сражение, крупнейшее в истории русского флота) под командованием адмиралов Спиридова и Орлова. 

Турецкий флот был уничтожен благодаря применению брандера "--- небольшого деревянного судна, заполненного чем"=то горючим, скрытно подгонявшегося к кораблям противника, прикреплявшегося к ним и подрывавшегося специально обученными людьми.

Генерал Румянцев одержал победу в битвах при Ланге и у местечка Кагул, причём последнее "--- торжество регулярной армии (27 тысяч человек) над нерегулярной (150 тысяч человек и 150 орудий). 

За счёт обученности и сплочённости русских войск турки были разгромлены.

Был захвачен Крым.

Эпидемия чумы и Пугачёвский бунт мотивировали Турцию продолжать войну, но позднее обречённость положения стала очевидна для турок. Был заключен Кючук"=Кайнарджирский мир 1774 года.

\textbf{По его условиям:}

\begin{itemize}
    \item{ Россия получила Крым и Кубань, междуречье Днепра и Южного Буга. Был заложен город Херсон. В Азове и Таганроге началось строительство флота. Керченский пролив также перешёл под контроль России }
    \item{ Турция обязалась свободно пропускать русские торговые корабли через проливы Босфор и Дарданелла }
    \item{ Крымское ханство получило политическую независимость от Османской империи }
\end{itemize}



Российское правительство начало активное освоение территории Северного Причерноморья. 

Именно тогда появляется в геополитической терминологии слово Новороссия. 

Здесь основнываются новые города "--- Екатеринослав, Херсон и некоторые другие.

Формально независимое Крымское ханство находится под оккупацией России до 1783 года.

Впоследствии Крымское ханство было присоединено к России. 

Был основан Севастополь.

Возникает <<греческий проект>>: Екатерина II, правящие верхи России и Потёмкин мечтают о восстановлении Византии. 

Активно строится Черноморский флот, осваивается Новороссия.

Османская империя и Иран пытаются создать свои сферы влияния.

По Георгиевскому трактату 1773 года Картли"=Осетинское царство перешло под протекторат России, что вызвало напряжённость в отношениях и с Турцией, и с Ираном. Расширение присутствия России приводит к новой войне с Османской империей.

\subsection{Русско"=турецкая война 1787--1791 гг.}

Россия снова действует в союзе с Австрией. 

Турцию поддерживает Швеция.

Боевые действия на Балтике велись в 1787"=1788 гг. и привели к подписанию мира на условиях статус"=кво.

Русская армия берёт Очаков. 

В 1789 Суворов одержал две крупные победы "--- в битве при Фокшанах и на реке Рынник. 

В обоих сражениях Суворов разгромил турецкие войска и получил титул граф Суворов-Рынникский.

В 1790 году Суворов взял <<неприступную>> крепость Измаил. 

Русским флотом командовал адмирал Фёдор Фёдорович Ушаков. 

За вклад в борьбу с иноверцами и благочестивый образ жизни он был причислен к лику святых.

В 1791 году флот Ушакова одержал победу над турецким у мыса Калиакрия.

После этих успехов был заключён Ясский мир 1791 года. 

По его условиям Турция признавала присоединение всего того, что уже было присоединено, включая Крым, и междуречье Южного Буга и Днестра. 

Протекторат над Картли"=Осетинским царством также был признан. 

Суверенитет России над Кабардой был признан.

В Новороссию активно переселялись сербы, болгары, армяне, немцы, греки и многие другие.

\subsection{Речь Посполитая.}

К 1760"=м годам это государство стало слабым и уязвимым. 

Инициатива о разделе Речи Посполитой поступила от Пруссии и Австрии, позднее Екатерина приняла предложение. 

В результате трёх разделов Речь Посполитая полностью прекратила существование. 

Курляндия, Белруссия и Литва, а также правобережная Малороссия вошли в состав России. 

Здесь проживали белорусы, украинцы, латыши, поляки, были сильны польская и еврейская диаспоры.

\subsection{Внешняя политика Павла I.}

Во Франции происходит Великая Революция(1789г) , устанавливается республика. Это приводит к череде коалиционных войн против Франции. При Екатерине Россия в этом не участвовала.

С 1798 года Россия вступает в коалицию против Франции. В ней также состояли Великобритания, Австрия и Турция.

Павел взял под своё покровительство остров Мальту в Средиземном море, провозгласил себя великим магистром Мальтийского ордена. Ушаков одерживает несколько побед, в том числе берёт крепость Корфу. Войско под командованием Суворова совершило Итальянский поход, где ему сопутствовал успех. В Швейцарии русский корпус под командованием Римского"=Корсакова оказался под угрозой уничтожения, и тот обратился за помощью к Суворову.  Суворов совершил переход через Альпы с армией и обозом, но он не успел помочь Римскому"=Корсакову, который был разбит французами.

Готовится поход Российской армии в Индию, но реализации этого плана помешал дворцовый.





