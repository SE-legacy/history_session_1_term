\section{Вопрос 24. Внутренняя политика в России в царствование Николая I.}

\subsection{Николай I (1825--1855) }

Николай I усилил центральное управление страной через СЕИВК (Собственная его Императорского Величества канцелярия), существовала 1812--1917 гг. Всего 6 отделений (4 "--- канцелярия императрицы, 5 "--- крестьянский вопрос, 6 "--- управление закавказьем)

\begin{enumerate}
    \item{отделение занималось личными бумагами императора. }
    \item{отделение "--- кодификационное "--- занималось законодательством Российской империи с 1826  г. Под руководством Сперанского 2"=е отделение создало полный свод законов РИ и опубликовало его в 1832 г. За проделанную работу Сперанский получил орден Андрея первозванного}
    \item{отделение "--- политическая полиция "--- цель: политический надзор, цензура, проведение секретных операций против внутренних врагов. Всего в нем работало ~ 40--50 человек.}
\end{enumerate}

Глава 3"=го отделения "--- А. Х. Бенкендорф.

 Создан корпус жандармов в 1827. Глава "--- Л. В. Дубельт. Цель "--- открытый надзор.

\subsection{Теория официальной народности.}

Созданная министром народного просвещения Уваровым. Согласно ей Россия держится на трех вещах:

\begin{enumerate}
    \item{Самодержавие}
    \item{Православие}
    \item{Народность}
\end{enumerate}

1826 г. "--- цензурный “чугунный” устав. Надзор за печатью и СМИ. Очень жесткие правила.

1835 г. "--- новый университетский устав. Ограничил автономию университетов. Вводилась должность “попечитель” университета, который утверждает должности. Усиливается надзор над студентами.

\subsection{Крестьянский вопрос.}

1837--1841 гг. "--- реформа гос деревни Кисилева.

\begin{enumerate}
    \item{В 1837 г. создано министерство государственных имуществ.}
    \item{Элементы самоуправления в крестьянской общине}
    \item{Оброк теперь зависит от дохода хозяйства.}
\end{enumerate}

\subsection{Денежная реформа Канкрина.}

1839--1843 гг.

\begin{enumerate}
    \item{Введение серебряного рубля}
    \item{Курс серебряного рубля привязан к бумажному}
    \item{Замена ассигнаций кредитными билетами.}
\end{enumerate}
