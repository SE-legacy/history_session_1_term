\section{Вопрос 7. Монгольское нашествие и установление зависимости Руси от Орды. Экспансия с запада в середине XII в. Историческая роль Александра Невского.}

\subsection{Монгольское нашествие и установление зависимости Руси от Орды:}

\begin{enumerate}
    \item{ Темучин получает титул Чингисхана и после завоевания Китая и получения передовых военных технологий, отправляется на запад, в сторону Руси. Поход задумывался как разведывательный.}
    \item{ Первыми удар принимают половцы и зовут на помощь русских князей. Происходит первое столкновение "--- битва на Калке (1223 г.). Монгольская армия одерживает победу. Разведывательный поход завершается более чем успешно, следующий был делом времени.}
    \item{ Поход на русь (1237--1241 г.)
        \begin{enumerate}
            \item{ Возглавляет поход хан Батый. Начинает с северо-восточных княжеств.}
            \item{ Декабрь 1236 г. "--- битва за Рязань. Потом бой при Коломне}
            \item{ Февраль 1238 г. "--- битва за Владимир.}
            \item{ Март 1238 г. "--- битва при реке Сить. Встречается армия Юрия Всеволодовича и Батыя. Из-за несогласованных действий со стороны русских князей, армия Батыя наносит сокрушительное поражение русским войскам и рассыпается по русской земле, захватывая города.}
            \item{ 1239 г. "--- Захват Чернигова и Переяславля}
            \item{ 1240 г. "--- Захват и практически полное разрушение Киева}
        \end{enumerate}
    }
    \item{ После завоевания и разграбления Золотой ордой (которая расположилась на больших территориях к югу и юго-востоку от Руси), большей части руси, была введена система «выходов» (сбора дани). Русским князьям выдавались «ярлыки», позволявшие собирать дань с русских земель и являвшиеся гарантом поддержки от Золотой Орды в случае княжеских междоусобиц}
    \item{ Первым, кто получил ярлык на княжение являлся Ярослав Всеволодович (сын Всеволода Большое Гнездо)}
\end{enumerate}

\subsection{Итогами вторжения монголов стали:}
\begin{enumerate}
    \item{ Демографическая яма}
    \item{ Разрушение городов}
    \item{ Удар по развитию ремесла}
    \item{ Нарушение традиционных торговых связей}
\end{enumerate}

\subsection{Экспансия с запада. Историческая роль Александра Невского}

В середине XIII в.  На основе литовских и финно-угорских племён формируется Литовское княжество, претендовавшее на русские земли.
Кроме того, этот период знаменит крестовыми походами. Тевтонский орден и орден меченосцев сливаются в один и начинают войну с Литвой и Русью за новгородские земли.

\begin{enumerate}
    \item{ 1239 г. "--- Отвоевание у Литвы Смоленска Ярославом Всеволодовичем}
    \item{ 1240 г. "--- Шведы высаживаются на берегу Невы, близ Новгорода.  Александр Ярославович (сын Ярослава Всеволодовича), разбивает шведскую армию и получает за это прозвище «Невский»}
    \item{ 5 апреля 1242 г. "--- Битва на Чудском озере (Ледовое побоище). После него Ливонский орден 10 лет не предпринимал наступательные действия против Руси.}
\end{enumerate}

Впоследствии Александр Невский стал князем во Владимире и получил от отца ярлык на княжение.

Таким образом, историческая роль Александра Невского заключается в предотвращении экспансии с запада, укреплении северо-западных рубежей Руси.
