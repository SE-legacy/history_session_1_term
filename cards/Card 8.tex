\section{Вопрос 8. Объединение русских земель вокруг Москвы в конце XIII--середине XV в.}

Для изучения данного вопроса следует понимать положение Руси в тот период времени.

\subsection{Положение Руси в тот переоид времени:}

\begin{enumerate}
    \item{ Раздробленность}
    \item{ Упадок прежних политических центров}
    \item{ Литовское княжество присоединяет западные территории}
    \item{ Уплата дани Орде}
    \item{ Угасание торговых связей}
\end{enumerate}

На смену старым политическим центрам приходят новые "--- Великий Новгород, Тверь, Москва. Это обусловлено их географическим положением, миграцией населения из"=за нашествия Орды, политическим успехом князей тех регионов.

Первым князем Москвы становится младший сын Александра Невского "--- Даниил Александрович (1276--1303 гг.):

\subsection{Даниил Александрович (1276--1303 гг.):}

\begin{enumerate}
    \item{Присоединил Коломну, Переяславль-Залесский}
\end{enumerate}

\subsection{Юрий Данилович (1303--1325 гг.) "--- сын Даниила Александровича:}

\begin{enumerate}
    \item{ Присоединил Можайск}
    \item{ Заявил права на великокняжеский титул и начал за него борьбу с Михаилом Ярославичем (тверским князем)}
    \item{ Женился на дочери ордынского хана Узбека и оговорил тверского князя, который был впоследствии убит. Получает за это ярлык на княжение. Сын убитого князя совершает покушение на Юрия Даниловича и ярлык переходит в Тверь.}
\end{enumerate}

\subsection{Иван Данилович (Калита) (1325--1340 гг.) "--- младший брат московского князя Юрия Даниловича.}

\begin{enumerate}
    \item{ В 1327 г. В Твери вспыхивает антиордынское восстание. Калита помогает ордынцам в подавлении, за что возвращает в Москву ярлык.}
    \item{ Уговаривает Киевского митрополита переехать в Москву}
    \item{ Инициирует перестройку стен Кремля, в том числе с использованием каменного строительства}
    \item{ Активно скупал вотчины, в которые сажал своих сыновей для расширения экономического и политического влияния}
\end{enumerate}

(Следующие два князя не упоминаются в лекциях из-за скудного количества информации о них, тем не менее, я решил их включить, дабы не разрывать таймлайн)

\subsection{Семён Иванович (Гордый) (1340--1353 гг.) "--- сын Ивана Калиты}

\begin{enumerate}
    \item{ После смерти отца Ивана Калиты многие князья поехали к ордынскому хану с просьбой не выдавать ярлык преемнику московского князя. Тем не менее, Узбек-хан всё же решил выдать ярлык Семёну}
    \item{ Воевал с Новгородом за дань.}
    \item{ Умер, предположительно от чумы}
\end{enumerate}

\subsection{Иван Иванович (Красный) (1353--1359 гг.) "--- сын Ивана Калиты, преемник Семёна Ивановича.}

\subsection{Дмитрий Иванович (Донской) (1359--1389 гг.) "--- сын Ивана Красного, внук Ивана Калиты.}

\begin{enumerate}
    \item{ Пришёл к власти в 9 лет, до взросления большую часть работы осуществлял митрополит Алексий.}
    \item{ Благодаря накопленным средствам были построены белокаменные стены Кремля}
    \item{ Участвует в войне с Литовским княжеством (1368--1372 гг.). Причиной войны являлась борьба между Литовским и Московским княжествами за передел Русских земель.}
    \item{ В 1375 г. Осуществляет поход на Тверь}
    \item{ После незаконного прихода к власти Мамая в Золотой Орде в 1370-х, Дмитрий Иванович отказывается платить дань. Мамай готовит поход на Москву, о чём так же знает Московский князь и во всеоружии встречает монгольскую армию на Куликовом поле в 1380 г. Мамай разгромлен, во время бегства был убит.}
    \item{ На смену Мамаю приходит Тохтамыш, который в 1382 г. Предпринимает повторный поход на московские земли, грабит Москву и восстанавливает выплату дани.}
    \item{ Благодаря действиям Дмитрия Донского Русь теряет политическую зависимость от Орды.}
    \item{ Ярлык на княжение теперь передаётся по наследству.}
\end{enumerate}

\subsection{Василий I (1389--1425 гг.) "--- сын Дмитрия Донского:}

\begin{enumerate}
    \item{ Присоединяет Мордву}
    \item{ Предпринимает попытку прекращения выплаты дани (после поражения Орды от Тамерлана), но хан Едигей собирает поход и восстанавливает зависимость.}
\end{enumerate}

\subsection{Василий II (Тёмный) (1425--1462 гг.) "--- сын Василия I:}

\begin{enumerate}
    \item{ Пришёл к власти в 11 лет}
    \item{ Боролся за власть с Юрием Дмитриевичем (братом Василия I). Василий II был вынужден вместе с боярами поехать в Коломну. В процессе борьбы был ослеплён, за что и получил прозвище.}
\end{enumerate}
