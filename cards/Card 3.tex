\section{Вопрос 3. Деятельность первых Рюриковичей (Олег, Игорь, Ольга, Святослав).}

\subsection{Олег (879--912 гг.)}

\begin{enumerate}
    \item {Перенёс столицу в Киев в 882 году благодаря походу}
    \item {Подчинил полян, северян, радимичей.}
    \item {В 907 г. В результате похода на Константинополь заключен первый для Русского гос"=ва договор о беспошлинной торговле.}
\end{enumerate}

\subsection{Игорь (912--945 гг.)}

\begin{enumerate}
    \item{Подчинял племена}
    \item{Введена регулярная система сбора дани "--- полюдье}
    \item{941 и 944 г. "--- Походы на Константинополь (удачный только второй)}
\end{enumerate}

\subsection{Ольга (945--964 гг.)}

\begin{enumerate}
    \item{ Введена система погостей"=уроков (погости "--- пункты, по которым взималась дань, уроки "--- размер дани).}
    \item{ Приняла христианство по восточному образцу после визита в Византию.}
    \item{ Жена Игоря}
\end{enumerate}

\subsection{Святослав (964--972 гг.)}

\begin{enumerate}
    \item{Сын Ольги и Игоря}
    \item{Разгром Волжской Болгарии и Хазарского каганата.}
    \item{Война с Византией:

        \begin{enumerate}
            \item{Занял Нижний Дунай}
            \item{После наступления Византии вынужден был отступить на Дунай}
            \item{Из"=за истощения войск был вынужден заключить мир и уступить все завоевания на Дунае}
            \item{Во время возвращения в Киев был убит подкупленными Византией печенегами}
        \end{enumerate}
    }
\end{enumerate}
