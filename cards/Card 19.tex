\section{Вопрос 19. Внутренняя политика Екатерины II и Павла І.}

\subsection{Личность Екатерины II.}

Екатерина II предстала перед обществом в хорошем свете:

\begin{itemize}
    \item{ Говорила и писала на русском языке }
    \item{ Вела себя как человек, истинно принявший православие }
    \item{ Всегда думала, прежде чем говорить, и стремилась всем понравиться.  }
    \item{ Старалась очаровать любого человека, с которым вступала в беседу }
\end{itemize}

\subsection{Идеология абсолютизма.}

В течение своего правления Екатерина Вторая стремилась соблюдать три главных принципа просвещенного абсолютизма:

\begin{itemize}
    \item{ Право монарха быть покровителем своего народа }
    \item{ Глава государства должен заниматься своим саморазвитием }
    \item{ Государь должен заботиться о просвещении и образовании подданных. }
\end{itemize}

\subsection{Мероприятия первых лет царствования.}

Новая правительница предприняла меры:

\begin{enumerate}
    \item{ Она всячески постаралась отмежеваться от убийства мужа. Была проведена мощная пропагандистская кампания против покойного императора с перечислением его истинных и мнимых грехов. }
    \item{ Она немедля попыталась обезопасить себя от Ивана Антоновича, так как среди некоторых гвардейцев возникли настроения в пользу шлиссельбургского арестанта. В Шлиссельбургскую крепость был отдан приказ усилить охрану узника и уничтожить его при малейшей попытке освобождения. И такой момент наступил в 1764 г. Поручик Смоленского полка, который охранял в те годы Шлиссельбургскую крепость, Мирович, следуя традициям дворцовых переворотов, попытался спасти Ивана Антоновича. Иван Антонович был убит: стражники выполнили приказ царицы. Переворот не удался. Мирович был схвачен и казнен. Теперь последний легитимный правитель, кроме 10"=летнего сына императрицы, был устранен. }
    \item{ Она подчинила все армейские части своим сторонникам, отменила нововведения в армии Петра III, копировавшие прусские воинские порядки, раздражавшие солдат и офицеров. }
    \item{ Она утвердила заготовленный Петром III указ о ликвидации Тайной канцелярии. }
\end{enumerate}

\subsection{Создание колоний иностранных поселенцев.}

25 октября 1762 года манифест «О позволении иностранцам селиться в России и свободном возвращении русских людей, бежавших за границу». (Екатерина II повелела заселить малообжитые районы России иностранцами)

\subsection{Уложенная комиссия.}

В 1767 г. Екатерина II, как и её предшественники, предприняла попытку создать новое Уложение, новый свод законов. Он должен был заменить Уложение 1649 г., учесть новые указы и манифесты, отразить изменившуюся жизнь страны.

Для разработки нового Уложения она решила созвать депутатов от разных сословий и разных мест России. 

Сам этот факт имел для самодержавной России огромное значение.

Вскоре в Москву, где открылось собрание, съехались более 500 депутатов. 

Больше всего в так называемой Уложенной комиссии было представителей городов, на втором месте стояло дворянство.

\subsection{Наказ Екатерины II.}

Наказ обосновывал политические принципы абсолютистского государства: власть монарха, сословное деление общества. Монархия признавалась идеальной формой правления, объявлялась источником неограниченной власти. Подразумевалось наличие и так называемой «власти средней», подчинённой монарху и помогающей ему управлять обществом. Роль монарха надзирать за их деятельностью. По мнению императрицы, абсолютная власть существует не для того, чтобы отнять у людей свободу, а для того, чтобы направлять их действия на достижение благой цели.

\subsubsection{Понятие свободы}

Свобода — право делать то, что разрешено законом.

\subsubsection{Сословная структура общества}

Сословная структура соотносилась с «естественным» делением общества на тех, кто по праву рождения может (и должен) повелевать и тех, кто призван с благодарностью принимать заботу правящего слоя. Помимо дворянства и «нижнего рода людей», то есть крестьян, существовал ещё и «средний род», то есть мещане. Отмена сословного неравенства в обществе, по мнению Екатерины, губительна и совершенно не подходит для русского народа.

\subsubsection{Понятие закона}
Закон — главный инструмент управления

По примеру Фридриха Великого, Екатерина II желала видеть в подвластном ей государстве торжество Закона. Закон рассматривался ею, как главный инструмент государственного управления, который необходимо сообразовывать с «духом народа», иначе говоря, с менталитетом. Закон должен обеспечивать полное и сознательное выполнение. Екатерина отмечала, что все сословия обязаны одинаково отвечать по уголовным преступлениям.

\subsubsection{Финансы и бюджет}

В Дополнении к «Наказу» 1768 года была проанализирована система финансового управления, перечислялись основные цели государства в этой сфере. Финансы должны были обеспечить «общую пользу» и «великолепие престола». Для решения этих задач требовалась правильная организация государственного бюджета.

\subsubsection{Уголовное право}

Касаясь уголовного права, Екатерина отмечала, что гораздо лучше предупредить преступление, нежели наказывать преступника. В наказе отмечалось, что нет необходимости наказывать голый умысел, не причинивший реального вреда обществу. Впервые в российском законодательстве была озвучена мысль о гуманистических целях наказания: об исправлении личности преступника. И только потом — о воспрепятствовании ему в дальнейшем причинять вред.  Наказание, согласно наказу, должно быть неизбежным и соразмерным преступлению.

\subsection{Крепостное право.}

1765 г. "--- указ разрешающий отправлять крестьян на каторгу

1767 г. "--- указ запрещающий крестьянам жаловаться на помещиков

1783 г. "--- указ вводивший крепостное право на Украине

Екатерина II попыталась стабилизировать обстановку щедрой раздачей наград и чинов. 

Одним из её методов привлечения на свою сторону знати, гвардии стало широкое предоставление дворянам государственных земель, населенных крестьянами. Сотни тысяч «крестьянских душ» превратились в крепостных.

\subsection{Восстание под руководством Пугачева.}

"Восстание Пугачёва" произошло в период с 1773 по 1775 годы в Российской империи. Емельян Пугачёв,возглавил крестьянское восстание, выступая против Екатерины II. Пугачёв заявлял, что он является правнуком Петра I и что царская власть теперь перешла к нему.
 
Он смог собрать вокруг себя значительный отряд, включая казаков, крестьян и других недовольных социальных групп. Восстание распространилось на большую часть Волжской и Уральской областей, вызвав серьезную тревогу в центре империи. Восстание Пугачёва стало одним из самых крупных в истории России и привело к значительному кровопролитию и разрушениям. В результате императорская армия долго боролась с повстанцами, и в 1775 году восстание было подавлено. 

Пугачёв был схвачен, казнен, а его приверженцы жестоко наказаны.

Восстание Пугачёва имело долгосрочные последствия для Российской империи. Оно привлекло внимание к проблемам социальной несправедливости и жестоких условий жизни крестьян. Власть ввела новые меры контроля над населением и ужесточила свои политические и социальные структуры в попытке предотвратить подобные восстания в будущем.

\subsection{Губернская реформа 1775 г.}

18 ноября 1775г императрицей Екатериной II было издано «Учреждение для управления губерний Российской империи», в соответствии с которым в 1775---1785 гг. была проведена кардинальная реформа административно"=территориального деления Российской империи. Задачей губернской реформы 1775 г. было укрепление власти дворянства на местах с целью предотвращения крестьянских восстаний.

\subsection{Жалованные грамоты дворянства и городам 1785г.}

В 1785 г. Екатериной II была издана Жалованная грамота дворянству, которая закрепила права дворян заниматься промышленной и торговой деятельностью, тем самым открыв для сословия новые перспективы деятельности. Жалованная грамота дворянству состояла из вводного манифеста и четырех разделов (92 статьи). В ней устанавливались принципы организации местного дворянского самоуправления, личные права дворян и порядок составления родословных дворянских книг.

Жалованная грамота дворянству представляла собой кодификацию законодательства о статусе дворянства. 

За дворянами закреплялись:

\textbf{личные права:}

\begin{enumerate}
    \item{ Телесная неприкосновенность (дворяне не подвергались телесным наказаниям и пыткам) }
    \item{ Право на геральдику (герб) }
    \item{ Освобождение от обязательной государственной службы, впервые утвержденное Петром III в Манифесте о вольности дворянству 1762 г. }
    \item{ Согласно названному Манифесту все, чем дворяне вознаграждались за службу, становилось их привилегиями}
\end{enumerate}

\textbf{имущественные права:}

\begin{enumerate}
    \item{ Монополия на обладание населенными имениями }
    \item{ Право на обладание недрами на помещичьей земле (в отличие от указа Петра I, который оставлял недра за государством) }
    \item{ Освобождение от податей и повинностей }
    \item{ Право на любую, не запрещенную законом предпринимательскую деятельность (кроме розничной торговли) }
    \item{ Винокуренная монополия. }
\end{enumerate}

На уездном и губернском уровнях создавались дворянские собрания, выбиравшие соответствующих предводителей дворянства. Дворяне выбирали своих сословных судей (для уездных судов и верхних земских судов) и даже часть чиновников.

\subsection{Внутренняя политика Павла I.}

Павел I правил Россией с 1796 по 1801 год. 

Павел I проводил ряд реформ:

\textbf{Административная реформа:}

Павел I ввел новую систему административного управления, создав новые ведомства и реорганизовав структуру правительства.

\textbf{Военная реформа:}

Под его правлением были введены новые военные законы, реорганизовано военное управление и проведены изменения в структуре армии.

\textbf{Экономические реформы:}

Павел I предпринял ряд мер по улучшению экономической ситуации в стране, включая развитие промышленности, сельского хозяйства и торговли.

\textbf{Культурные реформы:}

Под его правлением была проведена реформа образования, были созданы новые образовательные учреждения и развиты наука и культура.

\textbf{Политические реформы:}

Павел I внес изменения в систему управления, уменьшив влияние дворянства и введя новые политические институты.

Однако многие реформы Павла I были недостаточно успешными и вызвали недовольство в обществе, что способствовало его свержению и убийству. Его убийство в 1801 году было связано как раз с этими недовольствами и конфликтами.
