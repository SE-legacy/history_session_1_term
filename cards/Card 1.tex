\section{Вопрос 1. Восточные славяне в древности: происхождение, расселение, хозяйство, общественное развитие, религия.}

\subsection{Происхождение:}

Славяне относятся к индоевропейской языковой семье. Индоевропейцы обитали к югу от Кавказа, в нижнем Поволжье, на юге Сибири.

\subsection{Расселение:}

После Великого переселения народов в 4-7 вв. н.э. распались на:

\begin{enumerate}
    \item {Западных (Чехи, Поляки; Балитийское побережье и Европа)}
    \item {Южных (Сербы, Болгары; Бассейн Дуная + Балканы)}
    \item {Восточных (Русские, Белорусы; Верховье Волги, Ладога)}
\end{enumerate}

\subsection{Хозяйство:}

Наиболее подробно жизнь славян до образования Русского гос"=ва описывает <<Повесть временных лет>> Нестора. Основу хоз.деятельности составляло земледелие, славяне разводили скот, занимались рыболовством, охотничеством, бортничеством. Добывали железо, занимались кузнечным делом, гончарным (после появления в 7--8 вв. гончарных кругов), плотнецким делом.

\subsection{Общественное развитие:}

Родовые общины жили на берегу озера или реки. Старейшины были во главе общины. Несколько общин складывались в племена, главным политическим органом которых был совет старейшин. В 8--9 вв. начинается разрушение родо"=племенного строя и переход к соседскому строю (вожди с военно-ритуальными ф"=циями, сила военных), а затем к военной демократии.

\subsection{Религия:}

Язычество. Поклонение природным явлениям. Позднее "--- культ предков, в самом позднем этапе "--- вера в богов.

