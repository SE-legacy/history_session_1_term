\section{Вопрос 14. Россия в 1620-1680-е гг. Социально-экономическое развитие и внутренняя политика.}

\subsection{Этот период приходится на царствование трёх Романовых:}

\begin{enumerate}
    \item{ Михаила Федоровича (1613--1645 гг.) }
    \item{ Алексея Михайловича (1645--1676 гг.) }
    \item{ Федора Алексеевича (1676--1682 гг.) }
\end{enumerate}

Главной задачей стояло восстановление страны после разрушительных последствий Смутного времени.

\subsection{Михаила Федорович:}

\begin{enumerate}
    \item{ В 1620 г. создаются новые писцовые книги, в которые заносилось место жительства людей. }
    \item{ Начинается освоение чернозёмных территорий (снизу от Оки), Южной Сибири. }
    \item{ Появляются первые мануфактуры. Первая была основана в 1636 г. голландцем и была специализирована на металлургии. }
    \item{ Широкое распространение получают ярмарки (самые крупные "--- Макарьевская, Свенская, Ирбитская). В связи с этим возникает единый российский рынок (если цена падает в одном месте, то она падает везде) }
\end{enumerate}

\subsection{Алексей Михайлович:}

\begin{enumerate}
    \item{ В 1649 г. при Алексее Михайловиче принимается Соборное уложение, взамен устаревшего Судебника (1550 г.). В этом своде законов, среди прочего, были отменены урочные лета (срок сыска беглых крестьян). Это событие считается финальным в истории закрепощения крестьян. }
    \item{ В народном сознании складывается теория о самодержавии, как единственно верном способе правления. В связи с этим земские соборы постепенно утрачивают своё влияние и собираются всё реже. Последний земский собор был созван в 1653 г. На нём обсуждалось присоединение Украины. }
    \item{ В 1650-х возникает церковный раскол. Новые книги предлагалось переписать по древнерусским текстам (старообрядцы) или по греческим (новообрядцы во главе с Никоном). }
    \item{ В 1648 г. возникает Соляной бунт в Москве, вызванный недовольством повышения цен на соль. }
    \item{ В 1650 г. "--- Хлебный бунт в Новгороде и Пскове. }
    \item{ В 1662 г. "--- Медный бунт из-за выпуска новых медных денег, не вызывавших доверия населения }
    \item{ 1667 г. "--- Восстание Степана Разина:
        \begin{enumerate}
            \item{ Появившись в Астрахани, донское казачество под предводительством Степана Разина начало двигаться вверх по Волге, где увеличивали численность своей армии. }
            \item{ В конце 1670 г. Разин доходит до Симбирска, где его берут в осаду войска Алексея Михайловича. Войска Разина разбиты, он бежит на Дон, где его выдают царю. В 1671 г. Разин казнён. }
        \end{enumerate}
    }
\item{ В 1667 г. принимается Новоторговый устав, согласно которому иностранные купцы должны были платить пошлину для торговли в портовых городах и двойную пошлину, для торговли в непортовых городах. }
\end{enumerate}

\subsection{Федор Алексеевич:}

\begin{enumerate}
    \item{ В 1682 г. при Фёдоре Алексеевиче отменяется местничество (система распределения служебных мест) }
\end{enumerate}
