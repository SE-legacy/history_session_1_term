\section{Вопрос 10. Русское государство в XVI в.: внутренняя политика.}

\subsection{При Василии III (1505-1533 гг.):}

\begin{enumerate}
    \item{ Большая часть княжеских владений становится вотчинами. Князь был заинтересован в том, чтобы его приближённые получали земли из его рук. }
    \item{ Особой заботой князя являлась церковь, которая неуклонно наращивала свои богатства за счёт вкладов «на упокой души». Видя в церкви опору, он всё равно старался сделать рост богатств подконтрольным.}
    \item{ Активное развёртывание каменного строительства и пушкарного дела, к которым широко привлекаются иностранцы.}
\end{enumerate}

\subsection{При Елене Глинской (мать Ивана IV) (1534г-1538 г.):}

1535г.- Она фактически ввела в Русском государстве единую валюту (серебряная деньга).

\begin{enumerate}
    \item{ Глинская и боярский совет из семи вельмож, сформированный Василием. }
    \item{ Елена Глинская и ее фаворит Иван Оболенский быстро расправились с советом. }
    \item{ После смерти матери Ивана (1538) власть оказалась в руках боярской верхушки во главе с Василием и Иваном Шуйскими. }
    \item{ Вскоре в 1547г. по плану митрополиту Макария Ивана IV венчали первым российским царем.}
    \item{ (В стране на протяжении длительного периода наблюдалась нестабильность верховной власти, что привело к произволу местных феодалов. Это спровоцировало недовольства среди крестьянства и вылилось в открытые выступления в некоторых городах. Особенно резко социальные противоречия обострились в 1547 г. случился разрушительный пожар, который практически уничтожил столицу. Это стало главным поводом выступления московских посадских людей.)}
\end{enumerate}

\subsection{При Иване IV (Грозном) (1533 – 1584):}

\begin{enumerate}
    \item{ В 1547 г. По плану митрополита Макария, князь Иван IV принимает царский титул}
    \item{ Происходит созыв первого Земского собора (1549 г.). Помимо того, что на нём было принято решение о создании нового Судебника (1550 г.), этот созыв так же стал важным этапом к формированию сословно-представительной монархии.}
    \item{ В 1550 г. Формируется стрелецкий корпус. Определён единый порядок прохождения военной службы: «по отечеству» и «по прибору»}
    \item{  К существовавшим Дворцу (заведует великокняжескими землями) и Казне (финансовый центр и государственная канцелярия), добавляются приказы – отделившийся от казны орган, ведавший раздачей земель дворянам, их жалованием, ведущий учёт служилых людей). Их число постоянно растёт в связи с усложнением функций управления.}
\end{enumerate}

\textbf{Важным внутриполитическим событием в период правления Ивана IV (Грозного) стал период Опричнины (1565 – 1572 гг.):}

\begin{enumerate}
    \item{ В январе 1565 г. Уехав вместе с отобранными боярами и дворянами в Александрову слободу, Иван IV отправляет в Москву грамоту, в которой обвиняет церковных иерархов и всех землевладельцев в измене. Делегация московских бояр была вынуждена просить царя вернуться на трон. Тот согласился только при условии казнить и миловать по своему усмотрению. }
    \item{ Государство было поделено на две части: опричнину и земщину. Опричнина личным особым уделом царя, в который тот включил наиболее экономически развитые районы страны. На этих территориях поселялись дворяне, состоявшие в опричном войске, а бывшие владельцы выселялись в земщину. Опричное войско было верно царю и готово истребить любую измену в государстве, т.е. представляло из себя своеобразный карательный механизм.}
    \item{ К делам опричнины относят убийства многих церковных деятелей, массовые казни князей и дворян, Погром в Новгороде 1570 г., Московские репрессии 1570 г.}
    \item{ Конец опричнины был положен после того, как над государством нависла смертельная опасность в лице войска хана Девлет-Гирея. После того, как московские земли чуть не были потеряны, страна была вновь объединена.}
\end{enumerate}

\subsection{При Фёдоре Ивановиче (1584 – 1598 гг.)}

\begin{enumerate}
    \item{ Издаётся указ об урочных летах, согласно которому устанавливается пятилетний срок сыска беглых крестьян (впоследствии этот срок будет только увеличиваться)}
    \item{ В 1589 г. Вводится система патриаршества. Первым патриархом становится Иов.}
    \item{ В 1591 г. В Угличе умирает царевич Дмитрий – малолетний сын Ивана IV. А в 1598 г. Умирает бездетный Фёдор Иванович. Род Рюриков прекращает своё существование. На Земском соборе новым царём избран Борис Годунов.}
\end{enumerate}
