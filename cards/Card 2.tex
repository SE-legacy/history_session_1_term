\section{Вопрос 2. Образование древнерусского государства. <<Норманнская проблема>>.}

\subsection{Образование:}

Причинами образования древнерусского государства стали:

\begin{enumerate}
    \item{Развитие общественных отношений (разложение родо"=племенного строя), имущественное и социальное разложение.}
    \item{Возникновение конфликтов, способных решить только гос"=во. Племенные войны}
    \item{Необходимость противостоять внешней угрозе (Финно"=угорские племена, Хазарский каганат, набеги скандинавов)}
    \item{Необходимость масштабного обмена продукцией. Рост торговли => увеличение дистанции путей => потребность в безопасности. <<Из варяг в греки>>  подтолкнул к формированию гос"=ва}
\end{enumerate}

\subsection{Норманнская проблема:}

Суть проблемы:

\begin{enumerate}
    \item{Происхождение названия <<Русь>> (<<варяги>> "--- собирательное название скандинавских народов, а <<русь>> "--- один из таких народов; некое северно"=причерноморское племя, именуемое <<русью>>; название социального слоя, который впоследствии утратил своё первоначальное значение)}
    \item{Кто такой Рюрик? (Норманн или славянин?)}
    \item{Насколько велика роль варягов в образовании гос"=ва? (Настолько ли был немощным славянский народ, что не смог самостоятельно образовать государство и только лишь варяги смогли им помочь?)}
\end{enumerate}

Суть норманнской проблемы тесно связана с самой <<Легендой о призвании варяг>>, согласно которой в 9 веке славянские и финские племена платили дань варягам, а позднее отказались и изгнали их и стали править сами. Начались междоусобицы и потребовался князь, разрешивший бы их.

\subsection{Основные теории:}
\begin{enumerate}
    \item{Байер"=Миллер (<<Русь>>, <<Варяги>> "--- скандинавские слова, норманны участвовали в образовании гос"=ва)}
    \item{Ломоносов (<<Русь>>, <<Варяги>> "--- славянские или греческие слова, норманны не участвовали)}
\end{enumerate}
