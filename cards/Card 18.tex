\section{Вопрос 18. Внешняя политика России в конце XVII "--- первой половине XVIII в.}

\subsection{Причины Северной войны:}

\begin{enumerate}
    \item{ Противоречия между Швецией и странами Северного союза (Русское царство, королевство Дания, курфюршерство Саксония, королевство Речь Посполитая (король Август был одновременно королём Саксонии и Речи Посполитой одновременно)). }
    \item{ Дания, Саксония, Речь и Россия были недовольны господством Швеции на Балтийском море. }
    \item{ Россия стремилась вернуть утраченный по Столбовскому миру доступ к Балтике. }
\end{enumerate}

\subsection{Первый этап войны.}

Швецию возглавляет Карл XII, который сразу после начала войны переправил свою армию на территорию Дании. 

Он высадил своё войско близ Копенгагена, разбил датское войско и вынудил датского короля подписать сепаратный мир с Россией.

Россия осадила крепость Нарва.

Карл, освободившись в Дании, перебрасывает своё войско в Прибалтику и подходит к Нарве. 

В ноябре года русская армия при своём численном превосходстве (12 тысяч + гарнизон против петровских 34 тысяч солдат) претерпевает сокрушительное поражения. Дело в том, что регулярная армия Карла XII была прекрасно обучена и вооружена. Русская армия была организована из разнородных формирований под руководством бездарного полководца.

Карл XII подумал, что вывел второго противника из игры, и отправился воевать против польского короля. Карл XII воюет с Августом II. 

Пётр, как и в случае с Азовом не отчаялся, а стал готовиться к продолжению войны. Копировались шведские образцы, в будущем Пётр называл шведов учителями. Стали проводиться наборы солдатских контингентов, стала отливаться артиллерия. До войны поставщиком металла была Швеция, но теперь пришлось разрабатывать месторождения на Урале, а пока пришлось снять колокола и переплавить их на пушки. Командование армии Пётр поручил более грамотному полководцу Борису Петровичу Шереметьеву. 

В 1702 году взята крепость Нодебург, переименованная в Шлиссельбург (<<Ключград>>). 

Ниже по течению Невы была закончена следующая крепость. 

Нева оказалась в руках русской армии.

В мае 1703 года Пётр выплыл в устье Невы и решил на одном из многочисленных островов основать крепость под названием Санкт-Петербург. 

Эти успехи позволили заложить верфи на реке Свирь, а в 1704 году и напротив Санкт-Петербурга. 

Стал готовиться флот, чтобы использовать его против Швеции. 

Российские войска снова подошли к Нарве и Нарва была взята, а через несколько месяцев — ещё и крепость Дерпт (бывший Юрьев).

Когда Карл XII настиг войско Августа II, он разбил своего противника. 

На выборах нового короля был избран ставленник Карла XII. 

В 1706 году закончился первый этап Северной войны. 

\subsection{Второй этап войны.}

Россия осталась со Швецией один на один. 

Карл XII, направив свои силы на последнего соперника, в 1708 г. с территории Речи Посполитой продвинулся в Россию. Ожидалось, что он пойдёт на Москву, но Карл XII пошёл на юг, где ему обещал поддержку гетман Иван Мазепа, ранее сохранявший верность Петру I, но теперь, видимо, рассчитывающий на независимость. Туда же пошёл большой шведский отряд с огромным обозом: боеприпасами, продовольствием, медикаментами. 

России стало известно об этом, в 1708 г. у деревни Лесная русские войска вступили в схватку, одержали победу и захватили весь обоз, лишив Карла необходимого снабжения.

Битва у деревни Лесная стала матерью победы под Полтавой, как выразился сам Пётр I.

\subsection{Полтавская битва.}

Здесь столкнулись главные силы Петра и Карла XII. 

На стороне Петра было большинство, среди русских воинов в том числе были многие казаки, не поддержавшие Мазепу. 

У русского войска было превосходство и в артиллерии.

Войско Карла XII было разгромлено, значительная его часть оказалась в плену. 

Карл XII отправился на юг и скрылся на территории, принадлежащей Османской империи.

\subsection{Третий этап войны.}

Главных сухопутных сил Швеции больше не существовало, Россия стала доминировать на суши. 

Это изменило не только ход войны, но и статус России: новости о победе России показали, что самая мощная сухопутная армия в мире — русская.

\textbf{Восточная Прибалтика.} 

Здесь велись бои локального значения, главным препятствием остаётся шведский флот. 

Снова восстановился Северный союз: Август II (при поддержке России вернулся на трон), в союз вернулась и Дания.

Карл II, проживая в Стамбуле, убедил османского султана объявить России войну.

\subsection{Прутский поход 1710--1711гг. (неудачный).}

В 1710--1711 гг. состоялся Прутский поход, который оказался неудачен. 

Русская армия дошла до реки Прут, но здесь она на чуждой территории испытывала дефицит снабжения, совершила некоторые тактические ошибки. 

В этих условиях пришлось пойти на переговоры и удалось заключить мир: турецкая сторона тоже не желала вступать в генеральное сражение. 

Османам вернулся Азов, русский флот на Азовском море был ликвидирован. Это был унизительный мир для российской стороны, но теперь Россия могла сконцентрироваться на Прибалтике.

\subsection{Гангутское сражение 1714 г. (победа).}

Гангутское сражение 1714 г. состоялось у мыса Гангут. 

Русский флот состоял из галер, поэтому мог проходить на мелководье и был очень манёвренный. 

Используя эти преимущества над большими шведскими кораблями, Россия одержала победу.

Швеция была истощена, но английское правительство опасалось появления новой морской державы на Балтике и спонсировало Швецию.

\subsection{Победа у острова Грендган 1720г.}

В 1720 г. русский флот одержал ещё одну крупную победу у острова Грендган. 

Это сражение окончательно разгромило шведский флот, в результате которого русскому флоту открылись огромные возможности. 

Близ Стокгольма был высажен русский десант.

\subsection{1721 г. Нештадский мир и итоги войны.}

В Нештадте в 1721 г. был заключен мир. 

Россия закрепила за собой значительную часть восточной Прибалтики: Лифляндия, Эстляндия, побережье Финского залива и часть Карелии. 

Россия получила то, к чему стремилась изначально в этой войне, - удобный и большой выход к Балтийскому морю.

В ходе этой войны изменилось политическое положение. 

Её перестали воспринимать как большое слабое государство на задворках Европы. 

На постоянной основе в европейских столицах стали действовать русские послы, Романовы были связаны династическими браками с европейскими монархами. 

В ходе войны проводились военные, экономические, государственные и прочие реформы. 

После заключения Нештадского мира Сенат предложил принять Петру титул императора, Россия была провозглашена Империей.
