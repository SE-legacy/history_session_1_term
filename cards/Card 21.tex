\section{Вопрос 21. Внутренняя политика правительства при Александре I.}

\subsection{Личность Александра I.}

Император \textbf{Александр I (1801--1825)} взошел на престол вследствие военного переворота против Павла I. 

Александр был противоречивой натурой из"=за воспитания и курса правления. Александр был сыном Павла I. Его бабушка "--- Екатерина II очень любила старшего внука. Императрица после рождения забрала ребенка у отца с матерью и воспитывала его самостоятельно.

Одни современники отмечали его лицемерие и неискренность, другие приветливость, умение притягивать людей.
В начале правления Александр видел политическое и экономическое отставание России от передовых европейских государств, проводил либеральные реформы, но со временем превращался в консервативного и даже в последние годы жизни реакционного политика. 

Его глубокая религиозность, доходившая до мистицизма, отразилась в конкретных внутри- и внешнеполитических действиях в 1815--1825 гг

\subsection{Негласный комитет.}

Кружок старых друзей Александра I, сформировавшийся еще при правлении Павла I.

В него вошли молодые друзья царя "--- граф П. А. Строганов, польский князь А. Е. Чарторыйский, граф В. П. Кочубей и граф Н. Н. Новосильцев.

Проекты, которые они разрабатывали, не привели к конкретнным реформам. Дело ограничилось некоторыми частными преобразованиями.

Был расформирован в 1803 г.. Александр I уже уверенно сидел на престоле и не нуждался в советниках

\subsection{Реформа сената.}

В 1802 г. был реформирован Сенат, ставший высшим административным судебным и контролирующим органом в системе государственного управления. Его роль в законотворческой деятельности выражалась в том, что он получил право делать императору <<представления>> по поводу устаревших законов, а также участвовать в обсуждении новых.

(интересный факт "--- первая же попытка Сената возразить против указа 1803 г. об обязательной 12"=летней службе дворян, не достигших офицерского чина вызвала резкое "разъяснение" от Александра I, что Сенат может только делать <<представления>> ТОЛЬКО по поводу устаревших законов, а не будущих.)

\subsection{Утверждение министерств.}

Коллегии, созданные еще при Петре I утратили свою роль. В связи с этим в 1802 г. они были заменены министерствами.

В результате значительно усилилась центральная исполнительная власть.

Было учреждено 8 первых министерств: военно-сухопутных сил, военно-морских сил, иностранных дел, юстиции, внутренних дел, финансов, коммерции и народного просвещения.

\subsection{Крестьянский вопрос в политике Александра I.}

Еще во время коронации Александр торжественно заявил, что отныне прекращается раздача казенных крестьян в частные руки.

20 февраля 1803 г. "--- указ о вольных хлебопашцах предусматривал освобождение крепостных крестьян с землей за выкуп, целыми селениями или отдельными семействами, по обоюдной договоренности их с помещиком. Результаты указа ничтожные - 50 тыс чел ($0.5\%$ всех крестьяен) за правление Александра и 150 тыс до 1858 г.

Указы 1808 "--- 1809 гг. запрещали продавать крестьян на ярмарках, давать объявления в газетах о продаже дворовых, отменялось право помещиков ссылать по своей прихоти крепостных в Сибирь и т. д.

1818 г. "--- проект Аракчеева по освобождению крестьян, встретивший активное сопротивление помещиков.

\subsection{Преобразования в сфере народного просвещения.}

1802 г. "--- министерство Просвещения.

Устав 1804 г. для университетов, допускавшего их самоуправление. Однако позднее как проявление Аракчеевщины политика стала более консервативной и реакционной.

План государственного преобразования Сперанского и его реализация.

М. М. Сперанский разработал проект реформы государственного управления.

В его проекте "--- <<Введение к уложению государственных законов>> — намечался принцип разделения законодательной, исполнительной и судебной ветвей власти путем созыва представительной Государственной думы и введения выборных судебных инстанций. Все органы планировались лишь совещательными.

Единственный результат "--- учреждение Государственного совета в 1810 г., который состоял из министров и других высших чиновников, назначаемых императором. Сочетал в себе совещательные функции  и  распределение бюджета между министерствами.

Либеральный проект был встречен активным сопротивлением (в особенности Аракчеева), и как результат Сперанский был сослан в Сибирь.

\subsection{Внутренняя политика в 1815--1825 гг. А. А. Аракчеев.}

В последнее десятилетие правления Александра I во внутренней политике все больше ощущалась консервативная тенденция, получившая название "Аракчеевщина". политика нацелена на упрочнение Абсолютизма и крепостного права.

Один из печально известных проектов "--- военные поселения. Начали создаваться в 1815--1816 годах. Заключались в переселении военнообязанных крестьян и их семьи с целью самообеспечения армии. Проект провалился из-за недовольства армии и огромных трат.

Были и либеральные проекты, такие как введение конституции в Польше в 1820г.
