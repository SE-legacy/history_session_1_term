\section{Вопрос 17. Эпоха дворцовых переворотов.}

\subsection{Судьба престола после смерти Петра I.}

Пётр I был женат дважды, но с первой женой Пётр не ладил, они друг друга не любили, родился сын царевич Алексей. 

Потом Пётр женился на иноземке, крещённой как Екатерина Алексеевна в этом браке появились дочери.

Царевич Алексей оказался противником курса Петра, связался с враждебными государствами и был приговорён к смертной казни, но он умер в заточении.

В связи с отсутствием другого наследника Пётр издал Указ о престолонаследии (император мог назначить наследником любого православного Романова)

\subsection{Роль гвардии.}

В эпоху дворцовых переворотов гвардия становилась силой, возводившей на престол того или иного кандидата.

\subsection{Екатерина I и Меньшиков.}

Пётр I умирает. 

Формально не по решению Петра, но по традициям престолонаследия императорский титул должен был перейти семилетнему сыну царевича Алексея, Петру II.

Но генерал-фельдмаршал Меньшиков (возглавлявший Военную коллегию) с гвардией пришел в Зимний дворец и заявил, что императрицей будет жена Петра Екатерина, а Петр II становится наследником после Екатерины. 

Т. е. Меньшиков совершил первый дворцовый переворот. Он знал, что Екатерина будет править только со слов и с согласия Меньшикова. 

В этом были заинтересованы и другие довольно влиятельные люди. Екатерина могла помочь им сохранить и закрепить их новый статус.

\subsection{Екатерина I правила в период 1725--1727 гг.}

Она была малограмотной женщиной и не занималась государственными делами.  

При ней был создан верховный тайный совет, куда вошли Меньшиков и представители «новой знати». 

Этот совет принимал важнейшие решения, а остальные гос органы  эти решения выполняли.

При Екатерине были осуществлены реформы, которые не успел свершить Пётр I. 

В 1725 году была учреждена Российская академия наук. (Пётр готовил открытие академии, но помешало отсутствие учёных. Их пришлось приглашать из Европы. Первый состав РАН состоял полностью из иностранцев, которые должны были развивать в России науку.)

Экспедиция Витуса Беринга тоже была подготовлена Петром. 

Император поручил Витусу исследовать Дальний Восток и побережье Тихого океана и Беринг этим занимался. 

Беринг открыл Америку.

\subsection{Петр II и Долгорукие.}
В 1727 году Екатерина I умерла, к власти пришёл Пётр II (1727--1730). 

В момент воцарения Петру II было 11 лет, поэтому поначалу верховный тайный совет продолжил выполнять те же функции.

Меньшиков, чтобы закрепиться в своей роли фактического правителя, добивается для себя чина генералиссимуса, и планировал, что одна из его дочерей выйдет за Петра II, 

План провалился, тк среди верховного тайного совета, сената, генералитета и гвардии уже было много противников Меньшикова. 

Они составили заговор. 

Меньшиков был арестован, лишён чинов и званий, был отправлен в ссылку в Сибирь.

Теперь политическую жизнь России определяли Долгорукие и Голицыны, а также вице"=канцлер Андрей Иванович Остерман "--- внешнюю политику. 

Ближайший друг Петра II Иван Долгорукий, который был старше на несколько лет, весело проводил время с императором.

Государство начало отходить от реформаторской политики Петра I. 

Некоторые реформы сворачиваются, готовится свадьба Петра II с одной из княжен Долгоруких, но незадолго до свадьбы Пётр II умирает, от нездорового образа жизни.

\subsection{Анна Иоанновна и Э. И. Бирон.}

(Анна Иоанновна приходит к власти, попытка ограничения самодержавия: "верховники" "--- верховный совет, "кондиции" "--- условия прихода к власти от верховников)

Верховный тайный совет предложил престол дочери Ивана V Анне Иоанновне. (В своё время Пётр I использовал династические браки.)

Анна Иоанновна вышла замуж за герцога Курлядского. 

К Анне Иоанновне приехали посланцы, предложили стать российской императрицей. 

Она попросила условия (кондиции), и они были заготовлены заранее: верховный тайный совет сохраняется в своём составе и выполняет функцию реального управления государством. 

Императрица не может изменить состав совета, не может выйти замуж, без согласия верховного тайного совета не может назначить себе наследника.

Анна Иоанновна согласилась и приехала из Курляндии в Москву, где по традиции проходило миропомазание российских императоров. 

На эту коронацию съехалось дворянство, гвардия. 

Среди них распространились слухи о кондициях, и дворянство было очень недовольно этим. 

Анна Иоанновна, почувствовав за собой поддержку дворянства и гвардии, демонстративно надорвала бумагу с кондициями и была коронована в 1730 году. 

Долгорукие и Голицыны были отправлены в опалу (были лишены должностей и права появляться при дворе), а верховный тайный совет был распущен.

\subsection{Правление Анны Иоанновны}

Остерман "--- ближайший советник Анны Иоанновны.

Анна Иоанновна ввела 25"=летнее ограничение срока службы дворян(служба могла начинаться с момента рождения дворян). 

Соответственно, к 25--30 годам такие дворяне могли подать в отставку, если не хотели продолжать службу.

Крепостных теперь стало можно продавать отдельно от земли, даже разбивать семьи, т.е. превращало крепостных в подобие рабов.

Дворяне получили право создавать семьи из крепостных по своему усмотрению, отправить своих крепостных в Сибирь на каторжные работы за какие-либо провинности.

Анна Иоанновна не имела склонности к государственным делам и предпочитала проводить время в развлечениях. 

Государственные дела были поручены Бирону, Миниху и Остерману. 

У Анны Иоанновны было большое пристрастие к иностранцам: при покровительстве Бирона в Россию хлынули не слишком полезные для государства иностранцы, которые получали чины, поместья и т. д. Дворянам оставалось лишь терпеть подобную политику.

При Анне Иоанновне активно работала тайная канцелярия "--- политическая полиция, которая преследовала всех распространителей слухов о её отношениях с Бироном, критиков императрицы и её решений и т. д. Активно практиковались пытки не только против обвиняемых, но и против свидетелей.

Эта эпоха закончилась в 1740 году со смертью Анны Иоанновны. 

\subsection{Двойная смена власти}

Дворцовый переворот осуществили немцы, которые невзлюбили Бирона за его жёсткий и властный характер. 

Во главе стояли фельдмаршал Миних и канцлер Остерман. 

Бирон был арестован, лишён чинов и званий и сослан в Сибирь.  

В 1741  году гвардия возвела на престол Елизавету Петровну.

\subsection{Елизавета Петровна.}

\textbf{Личность Елизаветы Петровны}

Елизавета подчёркивала, что она не только дочь Петра I, но и продолжательница его дел, за счёт чего она и заручилась поддержкой гвардии.

В тюремное заточение отправился малолетний Иоанн Антонович, который всю свою жизнь проведёт в тюрьме. 

Его родители были отправлены в ссылку в Холмогоры.

Она отдавала предпочтение русскому дворянству. 

С иностранцами борьба не велась, многие из них сохранили свои должности. 

Были те, кто представлял опасность для императрицы. 

Но на высшие должности назначались русские. 

В частности, внешней политикой заведовал Бесстужин-Рюмин. 

Большую роль играли граф Воронцов и граф Пётр Шувалов, а также её фавориты — Алексей Разумовский, Иван Шувалов. 

Помимо личных отношений с императрицей они были важными государственными деятелями.

\subsection{Правление Елизаветы}

В области внутренней политики сохраняются привелегии дворянства, но при её правлении перестала применяться смертная казнь: преступников лишь ссылали на каторгу. Женщин освободили от телесных наказаний (клеймления, вырывания ноздрей, \ldots).  Семья преступника больше не была обязана отправляться в ссылку с преступником, хотя за ней сохранялось такое право. 

Внутренние таможенные пошлины были отменены, что стимулировало внутреннюю торговлю. 

Созданы купеческий банк и дворянский банк, дающий кредиты под залог имения.

При Елизавете началось генеральное межевание (межа "--- граница между земельными участками) земель. (стали создаваться карты, планы или письменные описания земельных участков с указанием их принадлежности)

При Елизавете в 1755 году возник Московский университет. (В организации и создании первого университета сыграл один из первых русских учёных Михаил Ломоносов.)

В 1756 году был создан первый российский театр как постоянно действующее учреждение(Александринский). Была основана академия художеств, президентом которой была назначена Елизавета Романовна Дашкова, в которой обучались музыканты, художники, скульпторы\ldots В Казани появилась первая гимназия.

\subsection{Петр III.}

У Елизаветы не было детей. Но у сестры Анны Петровны был сын. Ему было предложено стать наследником российского престола. Под именем Петра Фёдоровича он принял православие, женился на немецкой принцессе, Екатерине Алексеевне в православии. 

После смерти Елизаветы Петровны в 1761 году на престол взошёл Пётр III со своей женой Екатериной. 

Он правил всего полгода, с 1761 по лето 1762 г. 

Пётр III объявил о прекращении преследования раскольников, о начале секуляризации церковных земель (переводи земель из церковных в государственные). 

18 февраля 1762 г. "--- манифест о вольности дворянской. В нём объявилось, что дворяне могут не служить, если они этого не хотят, но они сохраняли за собой все привилегии

Тайная канцелярия была упразднена.

Казалось бы, такими мерами мог заслужить популярность среди дворянства, но он настроил против себя: гвардию, дворянство, Церковь, правящие круги. 

\subsubsection{Факторы, повлиявшие на негативное восприятие Петра III обществом:}
\begin{enumerate}
    \item{ Человек не русский и прибыл из"=за границы}
    \item{ вёл себя во многих ситуациях недостойно}
    \item{ презирал всё русское}
    \item{ стал отдавать предпочтение к иноземцам и окружая себя немцами, выражая презрение к русской армии (называя русских солдат янычарами), организовывая армию на прусский манер (притом, что Россия вела против Пруссии успешную войну).}
    \item{ Пётр III восхищался прусским лидером Фридрихом.}
\end{enumerate}
Против императора созрел заговор, составленный офицерами братьями Орловыми, Паниными, Дашковой, семьёй Воронцовых, которые планировали возвести на престол Екатерину.

\subsubsection{Факторы, повлиявшие на позитивное восприятие Екатерины обществом:}

\begin{enumerate}
    \item{ Говорила и писала на русском языке}
    \item{ Вела себя как человек, истинно принявший православие}
    \item{ Всегда думала, прежде чем говорить, и стремилась всем понравиться.}
\end{enumerate}

\subsection{Семилетняя война (1757--1763г).}

Крупнейшее событие европейской мировой истории

Главная причина "--- усиление Пруссии во главе с Фридрихом II. 

Пруссия была самым крупным по территории германским государством. 

Пруссия присоединяла к себе германские территории, создала себе мощнейшую отлично вооружённую армию.

Россия, Австрия и Франция создали союз против Пруссии. 

Англия заключила союз с Пруссией и поддерживала её в основном финансово. 

В 1756 г. Фридрих II нанёс поражение сначала Австрии, а затем и Франции. 

Первое крупное русско"=прусское сражение 1757 г. у Гросс"=Егерсдорфе.

Россия одержала победу и оккупировала восточную Пруссию, включая её главный город Кёнигсберг. (Правительство Елизаветы Петровны планировало сделать эту территорию частью Российской империи, поэтому местное население приводилось к присяге.)

Самое крупное сражение войны на территории Пруссии в 1758 г. "--- битва при Куннерсдорфе. 

Под командованием Салтыкова выступило русско"=австрийское войско, одержавшее победу в условиях артиллерийского паритета и малочисленности солдат. 

Русские войска в 1760 г. в первый раз взяли Берлин. 

Пётр III резко поменял направление внешней политики и перешёл к союзу с Фридрихом. 

Пруссии были возвращены все занятые территории, был заключён сепаратный мир. 

В союзе с Пруссией Пётр III планировал отправиться в поход против Дании, потому что датчане захватили его Родину — герцогство Гольштейн.

Этот неожиданный поворот стал главной причиной свержения Петра III.

Екатерина II сохранила в силе мир с Пруссией.

\textbf{Исторические последствия Семилетней войны для России:}

Франция и Австрия, заключая мир, закрепили небольшие пограничные пространства за собой. Русская армия подтвердила свой статус сильнейшей в Европе

\subsection{Переворот 28 июня 1762г.}

Пётр III был арестован и помещён в одной из резиденций неподалёку от Петербурга. 

На публике было объявлено, что он умер по естественным причинам, но на деле он был убит своей же охраной. 

Екатерина совершила переворот против своих мужа и сына Павла Петровича. 

Павел был объявлен наследником при Екатерине. 

Поручик Мирович в 1764 году попытался освободить Иоанна Антоновича и возвести его на престол, но охранник, в соответствии с данным на такой случай приказом, убил Иоанна Антоновича.

На этом эпоха дворцовых переворотов заканчивается.

