\section{Вопрос 6. Владимиро"=Суздальское, Новгородские и Галицко"=Волынские земли в начале XIII в.}

\subsection{Владимиро"=Суздальское:}

\begin{enumerate}
    \item{ Мало событий, почти ничего не происходило. Залесье – простонародное название из-за удалённости от Киева.}
    \item{ Юрий Долгорукий (1125--1157 гг.) "--- княжил в Суздале.

        \begin{enumerate}
            \item{ Получил прозвище из-за обширных сфер деятельности и, вероятно, из-за собирательства русских земель.}
            \item{ Стремился к княжению в Киеве (успешно в 1155 г.), отбирал земли у Новгородцев, воевал с Волжской Булгарией за беспошлинную торговлю по Волжскому торговому пути.}
            \item{ Основал Юрьев-Польский, Переяславль-Залесский, Дмитров. Считается основателем Москвы.}
        \end{enumerate}
    }
    \item{ Андрей Боголюбский (1157--1174 гг.) "--- сын Юрия Долгорукого.

        \begin{enumerate}
            \item{ Присоединил всё междуречье Оки и Волги}
            \item{ Стремился княжить в Киеве, но по достижению цели, отказался}
            \item{ Прозвище получил из-за набожности или частой жизни в селе Боголюбове}
            \item{ В 1169 г. Вместе с дружиной отправляется свергать Киевского князя, правившего от его лица.}
            \item{ Считается, что был очень жестоким}
            \item{ Убит в 1174 г. В результате боярского заговора}
        \end{enumerate}
    }
    \item{ Всеволод Юрьевич (Большое гнездо) (1176--1212 гг.) "--- Сын Юрия Долгорукого

        \begin{enumerate}
            \item{ После смерти Андрея Боголюбского разгорелась междоусобица между его племянниками от старшего брата Ярополком и Мстиславом и младшими братьями Михаилом и Всеволодом. Последнему удалось усмирить усобицы.}
            \item{ Прозвище получил из"=за большого потомства "--- 12 детей.}
            \item{ Продолжил строительство и процветание городов, основал Нижний Новгород}
            \item{ Был Великим Князем Киевским, но правил не из Киева.}
            \item{ В <<Слове о полку Игореве>> упоминается внушительная по численности дружина Владимира}
            \item{ Период правления считается периодом наивысшего могущества Владимирского княжества.}
            \item{ После смерти вновь начались междоусобицы его сыновей, что впоследствии ослабило эти земли.}
        \end{enumerate}
    }
\end{enumerate}

\subsection{Новгородские земли:}

\begin{enumerate}
    \item{ Вече в основном состояло из бояр}
    \item{ Князь выполнял судебную функцию для всех, кроме боярства}
    \item{ Из"=за территориального расположения нуждалось в привозе хлеба из Владимиро-Суздальского княжества, чем активно пользовались недоброжелатели. }
    \item{ Новгородские купцы укрепляли торговые связи с Ганзой "--- союзом северных немецких городов.}
    \item{ Высшей государственной должностью в Новгороде  был Посадник, который следил за деятельностью приглашённого князя, созывал вече, принимал послов.}
\end{enumerate}

\subsection{Галицко"=Волынские земли:}

\begin{enumerate}
    \item{ Находились в Прикарпатье. Являлись густонаселенной землёй.}
    \item{ Экономика выстроена на земледелии и добыче соли.}
    \item{ Ярослав Осмомысл (1153--1187 гг.) – князь Галицкий}
    \item{ Владел множеством языков, за что получил прозвище, означавшее <<Восемь умов>>}
\end{enumerate}
